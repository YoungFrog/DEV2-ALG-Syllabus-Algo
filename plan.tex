\documentclass[a4paper]{article}
\usepackage[utf8]{inputenc}
\usepackage[T1]{fontenc}
\usepackage{titlesec}
\renewcommand\thesection{\Roman{section}}
\titleformat{\section}[block]{\bfseries}{\thesection.}{1em}{}
\titleformat{\subsection}[block]{\hspace{2em}}{\thesubsection}{1em}{}
\titleformat{\subsubsection}[block]{\hspace{3em}}{\thesubsubsection}{1em}{}

\usepackage{array}

\begin{document}

\title{ALGORITHMIQUE II – Planning}
\maketitle

2e quadrimestre – 24 heures \\ 
(22 heures de cours + 2 heures d'interro formative)


\section{Tableaux à 2 dimensions (6 heures)}

\subsection{Définition}
\subsection{Notations}
\subsection{La troisième dimension}
\subsection{Parcours d'un tableau à 2 dimensions}
\subsection{Exercices}

\section{L'Orienté objet (4 heures)}
\subsection{Motivation}
\subsection{Illustration~: une durée}
\subsection{Quelques éléments de syntaxe}
\subsection{Mise en pratique~: le lièvre et la tortue}

INTERRO FORMATIVE~?

\section{La liste (6 heures)}

\subsection{La classe Liste}
\subsection{Comment implémenter l’état}
\subsection{Implémentation du comportement}
\subsection{Et sans tableau dynamique~?}
\subsection{Exercices}

\section{Représentation des données (6 heures)}

\subsection{Se poser les bonnes questions}
\subsection{Les structures de données}
\subsection{Exercices}

\newpage

\begin{tabular}{|l|l|}
	\hline
	Semaine num & Matière \\
	\hline
	1 &
	Tableaux 2D (2h)\\
	\hline
	2 &
	Tableaux 2D (2h)\\
	\hline
	3 &
	Tableaux 2D (2h)\\
	\hline
	4 &
	OO (2h)\\
	\hline
	5 &
	OO (2h)\\
	\hline
	6 &
	Interro formative (2h)\\
	\hline
	7 &
	Liste (2h)\\
	\hline
	8 &
	Liste (2h)\\
	\hline
	9 &
	Liste (2h)\\
	\hline
	10 &
	Représentation des données (2h)\\
	\hline
	11 &
	Représentation des données (2h)\\
	\hline
	12 &
	Représentation des données (2h)\\
	\hline
\end{tabular}

\end{document}