\chapter{Aide-mémoire}

Cet aide-mémoire peut vous accompagner lors d'une
interrogation ou d'un examen. Il vous est permis
d’utiliser ces classes et méthodes sans les développer.
Si vous sentez le besoin d’utiliser un objet ou une méthode qui
n'apparait pas ici, il faudra en écrire explicitement
le contenu et le code.

\section{Les caractères et les chaines}

\begin{Pseudocode}
	\Stmt {\large // Est-ce ?}
	\Empty
	\Stmt estLettre(car: caractère) \Gives~booléen		\RComment{est-ce une lettre ?}
	\Stmt estChiffre(car: caractère) \Gives~booléen		\RComment{est-ce un chiffre ?}
	\Stmt estMajuscule(car: caractère) \Gives~booléen	\RComment{est-ce une majuscule ?}
	\Stmt estMinuscule(car: caractère) \Gives~booléen	\RComment{est-ce une minuscule ?}
	\Empty
	\Stmt {\large // Conversions}
	\Empty
	\Stmt majuscule(car: caractère) \Gives~caractère	\RComment{convertit une minuscule en une majuscule.}
	\Stmt minuscule(car: caractère) \Gives~caractère	\RComment{convertit une majuscule en une minuscule.}
	\Stmt numLettre(car: caractère) \Gives~entier		\RComment{donne la position de la lettre dans l'alphabet.}
	\Stmt lettreMaj(n: entier) \Gives~caractère			\RComment{donne la lettre majuscule de position donnée.}
	\Stmt lettreMin(n: entier) \Gives~caractère			\RComment{donne la lettre minuscule de position donnée.}
	\Stmt chaine(car: caractère) \Gives~chaine			\RComment{convertit le caractère en une chaine.}
	\Let varChaine \Gets~varCaractère					\RComment{idem}
	\Stmt chaine(n : entier) \Gives~chaine				\RComment{convertit un entier en une chaine.}
	\Stmt chaine(x : réelr) \Gives~chaine				\RComment{convertit un réel en une chaine.}
	\Stmt nombre(ch : chaine) \Gives~réel				\RComment{convertit une chaine en un nombre.}
	\Empty
	\Stmt {\large // Manipulations}
	\Empty
	\Stmt longueur(ch : chaine) \Gives~entier			\RComment{donne la taille de la chaine.}
	\Stmt car(ch: chaine, n: entier) \Gives~caractère	\RComment{donne le caractère à une position donnée.}
	\Stmt sousChaine(ch: chaine, pos: entier, long: entier) \Gives~chaine \RComment{extrait une sous-chaine}
	\Stmt estDansChaine(ch: chaine, sous-chaine: chaine [ou caractère]) \Gives~entier 
	\Stmt 	\RComment{dit où commence une sous-chaine dans une chaine donnée (0 si pas trouvé)}
	\Stmt concat(ch1, ch2, \dots, chN: chaine) \Gives~chaine 	\RComment{concatène des chaines}
	\Let  ch \Gets~ch1 + ch2 + \dots + chN						\RComment{idem}
\end{Pseudocode}

\section{La liste}

\begin{Pseudocode}
\Class{Liste de T}
\Public
	\ConstrSign{Liste de T}{} 
	\MethodSign{get}{pos : entier}{T} \RComment retourne l’élément en position pos
	\MethodSign{set}{pos : entier, valeur : T}{} \RComment modifie l’élément en position pos
	\MethodSign{taille}{}{entier} \RComment retourne la taille de la liste
	\MethodSign{ajouter}{valeur : T}{} \RComment ajoute une valeur en fin de liste
	\MethodSign{insérer}{pos : entier, valeur : T}{} \RComment insère un élément en position pos
	\MethodSign{supprimer}{}{} \RComment supprime le dernier élément
	\MethodSign{supprimerPos}{pos : entier}{} \RComment supprime l’élément en position pos
	\MethodSign{supprimer}{valeur : T}{booléen} \RComment supprime l’élément de valeur donnée
	\MethodSign{vider}{}{} \RComment vide la liste
	\MethodSign{estVide}{}{booléen} \RComment indique si la liste est vide
	\MethodSign{existe}{valeur\In : T, pos\Out : entier}{booléen} \RComment recherche un élément
\EndClass
\end{Pseudocode}

\section{Date, Moment, Durée}

\begin{Pseudocode}
\Class{Date}
\Public
	\ConstrSign{Date}{} \RComment Crée la date du jour
	\ConstrSign{Date}{j, m, a : entiers} 
	\MethodSign{getJour}{}{entier}
	\MethodSign{getMois}{}{entier}
	\MethodSign{getAnnée}{}{entier}
	\MethodSign{égale}{autreDate : Date}{booléen}
	\MethodSign{estAntérieure}{autreDate : Date}{booléen}
\EndClass
\end{Pseudocode}

\begin{Pseudocode}
\Class{Moment}
\Public
	\ConstrSign{Moment}{} \RComment Crée le moment courant
	\ConstrSign{Moment}{h, m, s : entiers} 
	\MethodSign{getHeure}{}{entier}
	\MethodSign{getMinute}{}{entier}
	\MethodSign{getSeconde}{}{entier}
	\MethodSign{setHeure}{h : entier}{}
	\MethodSign{setMinute}{m : entier}{}
	\MethodSign{setSeconde}{s : entier}{}
	\MethodSign{égal}{autreMoment : Moment}{booléen}
	\MethodSign{estAntérieur}{autreMoment : Moment}{booléen}
\EndClass
\end{Pseudocode}

\begin{Pseudocode}
\Class{Durée}
\Public
	\ConstrSign{Durée}{secondes : entier} 
	\ConstrSign{Durée}{h, m, s : entiers} 
	\MethodSign{getJour}{}{entier}
	\MethodSign{getHeure}{}{entier}
	\MethodSign{getMinute}{}{entier}
	\MethodSign{getSeconde}{}{entier}
	\MethodSign{getTotalHeure}{}{entier}
	\MethodSign{getTotalMinute}{}{entier}
	\MethodSign{getTotalSeconde}{}{entier}
	\MethodSign{ajouter}{autreDurée : Durée}{}
	\MethodSign{soustraire}{autreDurée : Durée}{}
	\MethodSign{égale}{autreDurée : Durée}{booléen}
	\MethodSign{plusPetit}{autreDurée : Durée}{booléen}
\EndClass
\end{Pseudocode}


% Une méthode égale pour tous.
% ajouter valeurAbsolue(entier)
