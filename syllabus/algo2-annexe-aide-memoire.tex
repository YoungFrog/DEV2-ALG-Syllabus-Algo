\chapter{Aide mémoire}

	Cet aide-mémoire peut vous accompagner lors d'une
	interrogation ou d'un examen. 
	Il vous est permis d’utiliser ces méthodes sans les développer.
	Par contre, si vous sentez le besoin d’utiliser 
	une méthode qui n'apparait pas ici, 
	il faudra en écrire explicitement le contenu.

%===================================================
\section*{Manipuler les nombres}
%===================================================

	\begin{algo}
	\MethodSign{hasard}{}{réel}
		\RComment{Donne un nombre réel entre 0 inclus et 1 exclu.}
	\MethodSign{hasard}{\Par{min}{entier}, \Par{max}{entier}}{entier}
		\RComment{Donne un entier entre min et max inclus.}
	\end{algo}
	
	
% %===================================================
% \section*{Manipuler les chaines}
% %===================================================
	
% 	Remarque : lorsqu'on indique \lda{caractère},
% 	on signifie une chaine de longueur 1.
	
% 	\begin{description}
% 		\item[\lda{chaine[i]}]
% 		Désigne le i\ieme\ caractère de la chaine (en commençant à 1).
% 		\\Ex: \lda{texte[2] \Gets "a"}
% 		ou \lda{\K{afficher} texte[1]}
% 		\item[\lda{chaine1 + chaine2}]
% 		Produit une chaine qui est la concaténation des deux chaines.
% 		\item[\lda{long(chaine : chaine) \Gives{} entier}]
% 		Donne la longueur de la chaine (nb de caractères).
% 		\item[\lda{estLettre(car: caractère)\Gives~booléen}]
% 		Cette fonction indique si un caractère est une lettre. 
% 		Par exemple elle retourne vrai pour "a", "e", "G", "K", 
% 		mais faux pour "4", "\$", "@"\dots %$ 
% 		\item[\lda{estMinuscule(car: caractère)\Gives~booléen}]	
% 		Permet de savoir si le caractère est une lettre minuscule.
% 		\item[\lda{estMajuscule(car: caractère)\Gives~booléen}]	
% 		Permet de savoir si le caractère est une lettre majuscule.
% 		\item[\lda{estChiffre(car: caractère)\Gives~booléen}]	
% 		Permet de savoir si un caractère est un chiffre. 
% 		Elle retourne vrai uniquement pour les dix caractères 
% 		"0", "1", "2", "3", "4", "5", "6", "7", "8" et "9" 
% 		et faux dans tous les autres cas.
% 		\item[\lda{majuscule(texte: chaine)\Gives~chaine}]
% 		Retourne une chaine où toutes les lettres du texte
% 		ont été converties en majuscules.
% 		\item[\lda{minuscule(texte: chaine)\Gives~chaine}]
% 		Retourne une chaine où toutes les lettres du texte
% 		ont été converties en minuscules.
% 		\item[\lda{numLettre(car: caractère)\Gives~entier}]
% 		Retourne toujours un entier entre 1 et 26. 
% 		Par exemple \lda{numLettre("E")} donnera 5, 
% 		ainsi que \lda{numLettre("e")}. 
% 		Cette fonction traite donc de la même manière 
% 		les majuscules et les minuscules. 
% 		\lda{numLettre} retournera aussi 5 pour les caractères "é", "è", "ê", "ë"\dots). 
% 		Attention, il est interdit d’utiliser cette fonction 
% 		si le caractère n’est pas une lettre~!
% 		\item[\lda{lettreMaj(n: entier)\Gives~caractère}]
% 		Retourne la forme majuscule de la n\ieme{} lettre de l’alphabet 
% 		(où \textit{n} sera obligatoirement compris entre 1 et 26). 
% 		Par exemple, \lda{lettreMaj(13)} retourne "M".
% 		\item[\lda{lettreMin(n: entier)\Gives~caractère}]
% 		Idem pour les minuscules.
% 		\item[\lda{chaine(n: réel)\Gives~chaine}]
% 		Transforme un nombre en chaine.
% 		Ex: \lda{chaine(42)} retourne la chaine "42"
% 		et \lda{chaine (3,14)} donnera "3,14". 
% 		\item[\lda{nombre(ch : chaine) \Gives~réel}]
% 		Transforme une chaine contenant des caractères numériques 
% 		en nombre.
% 		Ainsi, \lda{nombre("3,14")} retournera 3,14. 
% 		C’est une erreur de l’utiliser avec une chaine
% 		qui ne représente pas un nombre.
% 		\item[\lda{sousChaine(ch: chaine, pos: entier, long: entier) \Gives~chaine}]
% 		Permet d’extraire une portion 
% 		d’une certaine longueur d’une chaine donnée, 
% 		et ceci à partir d’une position donnée. 
% 		\item[\lda{position(ch: chaine, sous-chaine: chaine)\Gives~entier}]	
% 		Permet de savoir 
% 		si une sous-chaine donnée 
% 		est présente dans une chaine donnée. 
% 		Elle permet d’éviter d’écrire 
% 		le code correspondant à une recherche. 
% 		La valeur de l’entier renvoyé est la position 
% 		où commence la sous-chaine recherchée. 
% 		Par exemple, 
% 		\lda{position("algorithmique", "mi")} retournera 9. 
% 		Si la sous-chaine ne s’y trouve pas, 
% 		la fonction retourne 0. 
% 	\end{description}
	
%===================================================
\section*{La liste}
%===================================================
	
	\begin{algo}
		\Comment{T est un type quelconque}\;
		\Class{\List{T}}{	
			\Public{
				\ConstrSign{\List{T}}{}			
					\RComment{construit une liste vide}
				\MethodSign{get}{pos : integer}{T}
					\RComment{donne un élément en position pos}
				\MethodSign{set}{pos : integer, valeur : T}{}
					\RComment{modifie un élément en position pos}
				\MethodSign{taille}{}{integer}
					\RComment{donne le nombre actuel d’éléments}
				\MethodSign{ajouter}{valeur : T}{}
					\RComment{ajoute un élément en fin de liste}
				\MethodSign{insérer}{pos : integer, valeur : T}{}
					\RComment{insère un élément en position pos}
				\MethodSign{supprimer}{}{}
					\RComment{supprime le dernier élément}
				\MethodSign{supprimerPos}{pos : integer}{}
					\RComment{supprime l'élément en position pos}
				\MethodSign{supprimer}{valeur : T}{boolean}
					\RComment{supprime l'élément de valeur donnée}
				\MethodSign{vider}{}{}
					\RComment{vide la liste}
				\MethodSign{estVide}{}{boolean}
					\RComment{la liste est-elle vide ?}
				\MethodSign{existe}{valeur \In : T, pos \Out : entier}{boolean}
					\RComment{recherche un élément}
			}
		}
	\end{algo}
