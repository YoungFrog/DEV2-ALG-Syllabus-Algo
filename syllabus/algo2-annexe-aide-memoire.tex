\chapter{Aide mémoire}

	Cet aide-mémoire peut vous accompagner lors d'une
	interrogation ou d'un examen. 
	Il vous est permis d’utiliser ces méthodes sans les développer.
	Par contre, si vous sentez le besoin d’utiliser 
	une méthode qui n'apparait pas ici, 
	il faudra en écrire explicitement le contenu.

%===================================================
\section*{Manipuler les nombres}
%===================================================

	\begin{algo}
	\MethodSign{hasard}{}{réel}
		\RComment{Donne un nombre réel entre 0 inclus et 1 exclu.}
	\MethodSign{hasard}{\Par{n}{entier}}{réel}
		\RComment{Donne un nombre entre entre 0 inclus et \lda{n} exclu.}
	\MethodSign{hasard}{\Par{min}{entier}, \Par{max}{entier}}{entier}
		\RComment{Donne un entier entre \lda{min} et \lda{max} inclus.}
	\end{algo}
		
%===================================================
\section*{La liste}
%===================================================
	
\begin{algo}
	\Comment{T est un type quelconque}\;
	\Class{\List{T}}{
		\Public{
			\ConstrSign{\List{T}}{}
			\RComment{construit une liste vide}
			\MethodSign{get}{pos : integer}{T}
			\RComment{donne un élément en position pos}
			\MethodSign{set}{pos : integer, valeur : T}{}
			\RComment{modifie un élément en position pos}
			\MethodSign{size}{}{integer}
			\RComment{donne le nombre actuel d’éléments}
			\MethodSign{isEmpty}{}{boolean}
			\RComment{la liste est-elle vide ?}
			\MethodSign{add}{value : T}{}
			\RComment{ajoute un élément en fin de liste}
			\MethodSign{add}{pos : integer, value : T}{}
			\RComment{insère un élément en position pos}
			\MethodSign{removePos}{pos : integer}{}
			\RComment{supprime l'élément en position pos}
			\MethodSign{remove}{value : T}{boolean}
			\RComment{supprime l'élément de valeur donnée}
			\MethodSign{clear}{}{}
			\RComment{vide la liste}
			\MethodSign{contains}{value : T}{boolean}
			\RComment{indique si un élément est présent}
			\MethodSign{indexOf}{value : T}{integer}
			\RComment{donne la position d'un élément}
		}
	}
\end{algo}
