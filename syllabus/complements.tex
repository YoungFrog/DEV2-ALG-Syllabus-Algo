\chapter{Compléments}

	Nous présentons ici quelques éléments qui pourront
	vous être utiles pour résoudre certains exercices.
	
%===================================================
\section{Énumération}
%===================================================

	Parfois, une variable ne peut prendre 
	qu’un ensemble fixe et fini de valeurs. 
	Par exemple une variable représentant une saison 
	ne peut prendre que quatre valeurs (HIVER, PRINTEMPS, ÉTÉ, AUTOMNE). 
	On va l’indiquer grâce à l’énumération 
	qui introduit un \textbf{nouveau type} de donnée.

	\begin{algo}
		\KwSty{enum} Saison \{ HIVER, PRINTEMPS, ÉTÉ, AUTOMNE \}
	\end{algo}

	Il y a deux avantages à cela~:~
	une indication claire des possibilités de la variable lors de la déclaration 
	et une lisibilité du code grâce à l’utilisation des valeurs explicites.

	Par exemple, 

	\begin{algo}
		\Comment{Lit une saison et affiche sa particularité}\;
		\Algo{particularitéSaisonnière}{}{}{
			\Decl{uneSaison}{Saison}\;
			\Read uneSaison
			\RComment{on lira la valeur HIVER ou PRINTEMPS ou ÉTÉ ou AUTOMNE}
			\uIf{uneSaison = HIVER}{
				\Write "il neige"\;
			}\uElseIf{uneSaison = PRINTEMPS}{
				\Write "les fleurs poussent"\;
			}\uElseIf{uneSaison = ÉTÉ}{
				\Write "le soleil brille"\;
			}\Else{
				\Write "les feuilles tombent"\;
			}
		}
	\end{algo}


	\begin{Exercice}{Autres situations}	
		Pouvez-vous identifier d’autres données qui pourraient
		avantageusement s’exprimer avec une énumération~?
	\end{Exercice}

	\subsection*{Quid des langages de programmation~?}
	%-------------------------------------------------
	
		Certains langages (comme Java) proposent un type énuméré complet.
		D’autres (comme C et C++) proposent un type énuméré incomplet 
		mais qui permet néanmoins une écriture comme celle ci-dessus.
		Cobol propose des «~noms conditions~» qui représentent
		l’ensemble des valeurs possibles d’une variable. 
		D’autres langages, enfin, ne proposent rien. 
		Pour ces langages, le truc est de définir des constantes entières 
		qui vont permettre une écriture proche de celle ci-dessus 
		(mais sans une déclaration explicite).

	\subsection*{Lien avec les entiers}
	%-------------------------------------
	
		Dans l’exemple ci-dessus, on lit une Saison mais souvent,
		si on travaille avec les Mois par exemple,
		on disposera plutôt d’un entier. 
		Il faut pouvoir convertir les valeurs. 
		Chaque langage de programmation propose sa propre technique; 
		nous allons adopter la syntaxe suivante~:

		\begin{algo}
			Saison(3) 
				\RComment{donne l’énumération de la saison numéro 3 (on commence à 1);}
				\RComment{donne ÉTÉ dans notre exemple.}
			position(uneSaison)
				\RComment{donne l’entier associé à une saison;}
				\RComment{si on a lu HIVER comme valeur pour uneSaison, donne la valeur 1.}
		\end{algo}
		
	
	%===================================================
	\section{Gestion des erreurs}
	%===================================================

		Lorsqu'un algorithme se trouve dans un état incorrect,
		particulièrement lorsqu'un paramètre est invalide,
		on peut l'indiquer via la primitive \lda{\Error}.
	
		Par exemple :	
		\begin{algo}
			\Algo{racineCarrée}{\Par{nb}{entier}}{}{
				\If{nb<0}{
					\Error{"Le nombre doit être positif"}\;
				}
				suite de l'algorithme\;
			}
		\end{algo}
			
		Pratiquement, cette primitive stoppe l'algorithme
		sans aucune possibilité de récupération.
		Dans un langage comme \bsc{Java} vous utiliserez
		le mécanisme des exceptions qui est plus souple. 