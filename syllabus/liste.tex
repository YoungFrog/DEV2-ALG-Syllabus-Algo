%=================
\chapter{La liste}
%=================

\marginicon{objectif}
Imaginons que l’on désire manipuler par programme une liste de contacts ou
encore une liste de rendez-vous. Cette liste va varier ; sa taille
n’est donc pas fixée. Utiliser un tableau à cet effet n’est pas l’idéal.
En effet, la taille d’un tableau ne peut plus changer une fois le tableau créé.
Il faudrait le surdimensionner, ce qui n’est pas économe.

Il serait intéressant de disposer d’une structure qui offre toutes les
facilités d’un tableau tout en pouvant «~grandir~» si nécessaire.
Construisons une telle structure de données et appelons-la «~List~»
pour rester en phase avec son appellation commune en Java.

Par exemple, considérons une liste de courses.
On pourrait la représenter ainsi :
\begin{enumerate}
	\item "fromage"
	\item "pain"
	\item "salami"
\end{enumerate}

On pourrait ajouter un élément en fin de liste, par exemple de l'eau,
pour obtenir la liste :

\begin{enumerate}
	\item "fromage"
	\item "pain"
	\item "salami"
	\item "eau"
\end{enumerate}

On pourrait aussi supprimer un élément de la liste, par exemple le pain,
et obtenir :

\begin{enumerate}
	\item "fromage"
	\item "salami"
	\item "eau"
\end{enumerate}

On pourrait aussi insérer un élément dans la liste,
par exemple une baguette,
ce qui décale, de facto, la position des suivants.

\begin{enumerate}
	\item "fromage"
	\item "salami"
	\item "baguette"
	\item "eau"
\end{enumerate}

Et encore plein de choses que nous allons détailler.

\newpage
%========================
\section{La classe List}
%=========================

Intéressons-nous au comportement qu’on attend d'une liste,
c'est-à-dire les méthodes qu’elle doit fournir.
Ce comportement sera identique quel que soit le type des éléments
de la liste; une liste de chaines et une liste d'entiers
ne se distinguent que par le type de certains paramètres
et valeurs de retour.
Ici, nous indiquons \lda{T} pour indiquer un type quelconque;
vous pouvez le remplacer par ce qui vous convient :
entier, chaine, Date\dots

Inspirons-nous de ce qui existe en \bsc{Java}.

\begin{algo}
	\Comment{T est un type quelconque}\;
	\Class{\List{T}}{
		\Public{
			\ConstrSign{\List{T}}{}
			\RComment{construit une liste vide}
			\MethodSign{get}{pos : integer}{T}
			\RComment{donne un élément en position pos}
			\MethodSign{set}{pos : integer, valeur : T}{}
			\RComment{modifie un élément en position pos}
			\MethodSign{size}{}{integer}
			\RComment{donne le nombre actuel d’éléments}
			\MethodSign{isEmpty}{}{boolean}
			\RComment{la liste est-elle vide ?}
			\MethodSign{add}{value : T}{}
			\RComment{ajoute un élément en fin de liste}
			\MethodSign{add}{pos : integer, value : T}{}
			\RComment{insère un élément en position pos}
			\MethodSign{removePos}{pos : integer}{}
			\RComment{supprime l'élément en position pos}
			\MethodSign{remove}{value : T}{boolean}
			\RComment{supprime l'élément de valeur donnée}
			\MethodSign{clear}{}{}
			\RComment{vide la liste}
			\MethodSign{contains}{value : T}{boolean}
			\RComment{indique si un élément est présent}
			\MethodSign{indexOf}{value : T}{integer}
			\RComment{donne la position d'un élément}
		}
	}
\end{algo}

Quelques précisions s’imposent :
\begin{itemize}
	\item
	      \lda{get(pos)} et \lda{set(pos,value)}
	      permettent de connaitre ou modifier un élément de la liste.
	      Comme pour les tableaux, le premier élément de la liste est en position 0.
	\item
	      \lda{add(value)}
	      ajoute un élément en fin de liste.
	      Elle grandit donc d’une unité.
	\item
	      \lda{add(pos,value)}
	      insère un élément à une position donnée (entre 0 et taille-1).
	      L’élément qui s’y trouvait est décalé d'une position
	      ainsi que tous les éléments suivants.
	\item
	      \lda{remove(value)}
	      enlève un élément de valeur donnée.
	      Elle retourne un booléen indiquant si la suppression a pu se faire ou pas
	      (si la valeur était présente ou pas).
	      Si la valeur existe en plusieurs exemplaires,
	      la méthode n’en supprime que la première	occurrence.
	\item
	      \lda{removePos(pos)}
	      supprime un élément d'une position donnée en décalant les éléments suivants.
	      Nous avons changé le nom de la version \bsc{Java} qui l'appelle
	      \samp{remove} pour ne pas confondre avec la précédente
	      dans le cas d'une liste d'entiers.
	\item
	      \lda{contains(value)}
	      permet de savoir si un élément donné existe dans la liste.
	\item
	      \lda{indexOf(value)}
	      donne la position d'un élément dans la liste.
	      \begin{itemize}
		      \item
		            si l’élément n’existe pas, elle retourne $-1$.
		      \item
		            si l’élément est présent en plusieurs exemplaires,
		            la méthode donne la position de la première occurrence.
	      \end{itemize}
	\item
	      En pratique, il serait intéressant de chercher un élément à partir d’une
	      partie de l’information qu’elle contient mais c’est difficile à
	      exprimer de façon générique c'est-à-dire lorsque le
		  type n'est pas connu à priori.
	\item 
		  Il existe encore plus de méthodes en \bsc{Java}.
		  Nous vous invitons à consulter la documentation :
		  \url{https://docs.oracle.com/en/java/javase/13/docs/api/java.base/java/util/List.html}
\end{itemize}

\clearpage

{\sffamily\bfseries\scshape
	Exemple : manipulations de base}
%-------------------------------

Soit l'algorithme suivant :

\begin{algo}
	\Algo{ex1}{}{}{
		\Decl{liste}{\List{integers}}\;
		liste \Gets \New \List{integers}()\;
		liste.add(42)\;
		liste.add(54)\;
		liste.set(1,44)\;
		liste.add(1,43)\;
		liste.removePos(2)\;
		liste.remove(42)\;
		liste.clear()\;
	}
\end{algo}

Après sa création, la liste est vide.
Ensuite, elle passe par les états suivants :

\begin{minipage}[t]{2cm}
	\begin{enumerate}[start=0]
		\item 42
	\end{enumerate}
\end{minipage}
\begin{minipage}[t]{2cm}
	\begin{enumerate}[start=0]
		\item 42
		\item 54
	\end{enumerate}
\end{minipage}
\begin{minipage}[t]{2cm}
	\begin{enumerate}[start=0]
		\item 42
		\item 44
	\end{enumerate}
\end{minipage}
\begin{minipage}[t]{2cm}
	\begin{enumerate}[start=0]
		\item 42
		\item 43
		\item 44
	\end{enumerate}
\end{minipage}
\begin{minipage}[t]{2cm}
	\begin{enumerate}[start=0]
		\item 42
		\item 43
	\end{enumerate}
\end{minipage}
\begin{minipage}[t]{2cm}
	\begin{enumerate}[start=0]
		\item 43
	\end{enumerate}
\end{minipage}

Enfin, le dernier appel la vide complètement

\bigskip
{\sffamily\bfseries\scshape
	Exemple : afficher une liste}
%-------------------------------

Écrivons l'algorithme qui affiche tous les éléments d'une liste,
reçue en paramètre.

\begin{algo}
	\Algo{afficher}{\Par{liste}{\List{integers}}}{}{
		\Fori{i}{0}{liste.taille()-1}{
			\Write liste.get(i)\;
		}
	}
\end{algo}

\bigskip
{\sffamily\bfseries\scshape
	Exemple : recherche du minimum}
%-------------------------------

Dans le chapitre sur les tableaux, vous avez fait un exercice consistant
à afficher tous les indices où se trouve le minimum d’un tableau.
Reprenons-le et modifions-le afin qu’il retourne la liste des indices
où se trouvent les différentes occurrences du minimum. On pourrait
l’écrire ainsi :

\begin{algo}
	\Algo{indicesMinimum}{\Par{tab}{\Array{n}{integers}}}{\List{integers}}{
		\Decl{min}{integer}\;
		\Decl{indicesMin}{List of integers}\;
		min \Gets tab[0]\;
		indicesMin \Gets \New \List{integers}()\;
		indicesMin.ajouter(0)\;
		\Fori{i}{1}{n-1}{
			\uIf{tab[i] = min}{
				indicesMin.ajouter(i)\;
			}\ElseIf{tab[i] < min}{
				indicesMin.vider() \;
				indicesMin.ajouter(i)\;
				min \Gets tab[i]\;
			}
			\Comment{rien à faire si tab[i] > min}\;
		}
		\Return indicesMin\;
	}
\end{algo}

\newpage
En \bsc{Java} ça donne :
\listing*[linerange=8-24]{java}{java/src/main/java/dev2/algo/liste/IndiceMin.java}

%==================
\section{Exercices}
%===================

\begin{Exercice}{Manipulation d'une liste}
	Écrire un algorithme qui crée la liste suivante :
	\begin{enumerate}[start=0]
		\item 494
		\item 209
		\item 425
	\end{enumerate}
	affiche sa taille, demande si la valeur 425 est présente,
	supprime la valeur 209 puis insère la valeur 101 en tête de liste.
\end{Exercice}

\begin{Exercice}{Liste des premiers entiers}
	Écrire un algorithme qui reçoit un entier $n$ en paramètre et retourne la
	liste contenant les entiers de 1 à $n$ dans l'ordre
	décroissant. On peut supposer que $n$ est strictement positif.
\end{Exercice}

\begin{Exercice}{Somme d'une liste}
	\marginicon{java}%
	Écrire un algorithme qui calcule la somme des éléments d’une liste
	d’entiers.
\end{Exercice}

\begin{Exercice}{Anniversaires}
	Écrire un algorithme qui reçoit une liste de structure Personne
	(nom + prénom + date de naissance) et retourne la liste
	de ceux qui sont nés durant un mois passé en paramètre
	(donné sous la forme d'un entier entre 1 et 12).
\end{Exercice}

\begin{Exercice}{Concaténation de deux listes}
	\marginicon{java}%
	Écrire un algorithme qui reçoit 2 listes et ajoute
	à la suite de la première les éléments de la seconde; la seconde liste
	n'est pas modifiée par cette opération.
\end{Exercice}

\begin{Exercice}{Le nettoyage}
	Écrire un algorithme qui reçoit une liste de chaines en paramètre et
	supprime de cette liste tous les éléments de valeur donnée en
	paramètre. L'algorithme retournera le nombre de
	suppressions effectuées.
\end{Exercice}

\newpage
\begin{Exercice}{Les extrêmes}
	\marginicon{java}%
	Écrire un algorithme qui supprime le minimum et le maximum des éléments
	d’une liste d’entiers. On peut supposer que le maximum et le minimum
	sont uniques.
\end{Exercice}

\begin{Exercice}{Fusion de deux listes}
	\marginicon{java}%
	Soit deux listes \textbf{triées}
	d'entiers (redondances possibles). Écrire un algorithme
	qui les fusionne. Le résultat est une liste encore triée contenant
	tous les entiers des deux listes de départ (qu'on
	laisse inchangées).

	Exemple : Si les 2 listes sont (1, 3, 7, 7) et (3, 9),
	le résultat est (1, 3, 3, 7, 7, 9).
\end{Exercice}

\begin{Exercice}{Éliminer les doublons d'une liste}
	\marginicon{java}%
	Soit une liste \textbf{triée}
	d'entiers avec de possibles redondances. Écrire un
	algorithme qui enlève les redondances de la liste.

	Exemple : Si la liste est (1, 3, 3, 7, 8, 8, 8),
	le résultat est (1, 3, 7, 8).

	\begin{enumerate}[label=\alph*)]
		\item
		      Faites l'exercice en créant une \textbf{nouvelle
			      liste} (la liste de départ reste inchangée)
		\item
		      Refaites l'exercice en \textbf{modifiant}
		      la liste de départ (pas de nouvelle liste)
	\end{enumerate}
\end{Exercice}

\begin{Exercice}{Rendez-vous}
	Soit la structure \lda{RendezVous} composée
	d’une date (cf. la structure Date du cours de DEV1)
	et d’un motif de rencontre.

	\begin{algo}
		\Structure{RendezVous}{
			\Decl{date}{Date}\;
			\Decl{motif}{string}\;
		}
	\end{algo}

	Écrire un algorithme qui reçoit une liste de rendez-vous
	et la met à jour en supprimant tous ceux qui sont désormais passés.

	Pour résoudre cet exercice, vous pouvez utiliser sans l'écrire
	un algorithme \lda{aujourdhui()} qui retourne la date du jour.
\end{Exercice}