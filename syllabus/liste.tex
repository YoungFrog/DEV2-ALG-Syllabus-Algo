%=================
\chapter{La liste}
%=================

	\marginicon{objectif}
	
	Imaginons qu’on désire manipuler par programme une liste de contacts ou
	encore une liste de rendez-vous. Cette liste va varier ; sa taille
	n’est donc pas fixée. Utiliser un tableau à cet effet n’est pas l’idéal. 
	En effet, la taille d’un tableau ne peut plus changer une fois le tableau créé. 
	Il faudrait le sur-dimensionner, ce qui n’est pas économe.
	
	Il serait intéressant de disposer d’une structure qui offre toutes les
	facilités d’un tableau tout en pouvant «~grandir~» si nécessaire.
	Construisons une telle structure de données et appelons-la «~Liste~»
	pour rester en phase avec son appellation commune en Java.
	
	Par exemple, considérons une liste de courses.
	On pourrait la représenter ainsi :
	\begin{enumerate}
	\item "fromage"
	\item "pain"
	\item "salami"
	\end{enumerate}
	
	On pourrait ajouter un élément en fin de liste, par exemple de l'eau,
	pour obtenir la liste :
	
	\begin{enumerate}
	\item "fromage"
	\item "pain"
	\item "salami"
	\item "eau"
	\end{enumerate}
	
	On pourrait aussi supprimer un élément de la liste, par exemple le pain,
	et obtenir :
	
	\begin{enumerate}
	\item "fromage"
	\item "salami"
	\item "eau"
	\end{enumerate}
	
	On pourrait aussi insérer un élément dans la liste, 
	par exemple une baguette, 
	ce qui décale, de facto, la position des suivants.
	
	\begin{enumerate}
	\item "fromage"
	\item "salami"
	\item "baguette"
	\item "eau"
	\end{enumerate}
	
	Et encore plein de choses que nous allons détailler.
	
	%========================
	\section{La classe Liste}
	%=========================
	
		Nous verrons plus loin comment réaliser une classe Liste en pratique 
		mais nous pouvons déjà définir le comportement qu’on en attend 
		(les méthodes qu’elle doit fournir)
		
		Ce comportement sera indentique quel que soit le type des éléments
		de la liste; une liste de chaine et une liste d'entiers
		ne se distinguent que par le type de certains paramètres
		et valeurs de retour.
		Ici, nous indiquons \lda{T} pour indiquer un type quelconque;
		vous pouvez le remplacer par ce qui vous convient : entier, chaine, Date\dots.
		
		\begin{LDA}
			\Class{Liste de T}
				\RComment T est un type quelconque
				\Private
					\LComment sera complété plus tard	
				\Public
					\ConstrSign{Liste de T}{}			
						\RComment construit une liste vide
					\MethodSign{get}{pos : entier}{T}
						\RComment donne un élément en position pos
					\MethodSign{set}{pos : entier, valeur : T}{}
						\RComment modifie un élément en position pos
					\MethodSign{taille}{}{entier}
						\RComment donne le nombre actuels d’éléments
					\MethodSign{ajouter}{valeur : T}{}
						\RComment ajoute un élément en fin de liste
					\MethodSign{insérer}{pos : entier, valeur : T}{}
						\RComment insère un élément en position pos
					\MethodSign{supprimer}{}{}
						\RComment supprime le dernier élément
					\MethodSign{supprimerPos}{pos : entier}{}
						\RComment supprime l'élément en position pos
					\MethodSign{supprimer}{valeur : T}{booléen}
						\RComment supprime l'élément de valeur donnée
					\MethodSign{vider}{}{}
						\RComment vide la liste
					\MethodSign{estVide}{}{booléen}
						\RComment la liste est-elle vide ?
					\MethodSign{existe}{valeur \In : T, pos \Out : entier}{booléen}
						\RComment recherche un élément
				\EndClass
		\end{LDA}
		
		\bigskip
		
		Quelques précisions s’imposent :
		\begin{itemize}
			\item 
				Les méthodes «\lda{~get~}» et «\lda{~set~}»
				permettent de connaitre ou modifier un élément de la liste. On
				considère, au cours d'algorithmique, que le premier élément de 
				la liste est en position 0.
			\item 
				«\lda{~ajouter~}» ajoute un élément en fin de liste (elle
				grandit donc d’une unité)
			\item 
				«\lda{~insérer~}» insère un élément à une position donnée
				(entre 1 et taille+1). L’élément qui s’y trouvait est décalé
				d'une position ainsi que tous les éléments suivants.
			\item 
				La méthode «\lda{~supprimerPos~}»
				supprime un élément d'une position donnée en
				décalant les éléments suivants. On pourrait imaginer une technique plus
				rapide consistant à placer le dernier élément à la place de l’élément
				supprimé mais ce faisant on changerait l’ordre relatif des éléments ce
				qui va à l’encontre de l’idée intuitive qu’on se fait d’une liste.
				Cette amélioration pourrait plutôt s’envisager dans une structure de
				type \textbf{ensemble} pour lequel il n’y a pas d’ordre relatif entre
				les éléments.
			\item 
				La version de «\lda{~supprimer~}» avec une valeur en
				paramètre enlève un élément de valeur donnée. Elle retourne un booléen
				indiquant si la suppression a pu se faire ou pas (ce qui sera le cas si
				la valeur n’est pas présente dans la liste). Si la valeur existe en
				plusieurs exemplaires, on prendra la convention arbitraire que
				la méthode n’en supprime que la première	occurrence.
			\item 
				La méthode «\lda{~existe~}» permet de savoir si un élément
				donné existe dans la liste. 
				\begin{itemize}
					\item 
						si c’est le cas, elle précise aussi sa position dans le paramètre sortant 
						\lda{pos}
					\item 
						si l’élément n’existe pas, ce paramètre est	indéterminé 
					\item 
						si l’élément est présent en plusieurs exemplaires, la méthode donne la
						position de la première occurrence.
				\end{itemize}
			\item 
				En pratique, il serait intéressant de chercher un élément à partir d’une
				partie de l’information qu’elle contient mais c’est difficile à
				exprimer de façon générique c'est-à-dire lorsque le
				type n'est pas connu à priori.
		\end{itemize}
		
		{\sffamily\bfseries\scshape
		Exemple : manipulations de base}
		
		Soit l'algorithme suivant :
		
		\begin{LDA}
			\Algo{ex1}{}{}
				\Decl{l}{Liste d'entiers}
				\Let l \Gets \K{nouvelle} Liste d'entiers()
				\Stmt l.ajouter(42)
				\Stmt l.ajouter(54)
				\Stmt l.set(1,44)
				\Stmt l.insérer(1,43)
				\Stmt l.supprimerPos(2)
				\Stmt l.supprimer(42)
				\Stmt l.vider
			\EndAlgo
		\end{LDA}
		
		Après sa création, la liste est vide.
		Ensuite, elle passe par les états suivants :
		
		\begin{minipage}[t]{2cm}
		\begin{enumerate}[start=0]
		\item 42
		\end{enumerate}
		\end{minipage}
		\begin{minipage}[t]{2cm}
		\begin{enumerate}[start=0]
		\item 42
		\item 54
		\end{enumerate}
		\end{minipage}
		\begin{minipage}[t]{2cm}
		\begin{enumerate}[start=0]
		\item 42
		\item 44
		\end{enumerate}
		\end{minipage}
		\begin{minipage}[t]{2cm}
		\begin{enumerate}[start=0]
		\item 42
		\item 43
		\item 44
		\end{enumerate}
		\end{minipage}
		\begin{minipage}[t]{2cm}
		\begin{enumerate}[start=0]
		\item 42
		\item 43
		\end{enumerate}
		\end{minipage}
		\begin{minipage}[t]{2cm}
		\begin{enumerate}[start=0]
		\item 43
		\end{enumerate}
		\end{minipage}
		
		Enfin, le dernier appel la vide complètement
		
		\bigskip
		{\sffamily\bfseries\scshape
		Exemple : recherche du minimum}
		
		Dans le chapitre sur les tableaux, vous avez fait un exercice consistant
		à afficher tous les indices où se trouve le minimum d’un tableau.
		Reprenons-le et modifions-le afin qu’il retourne la liste des indices
		où se trouvent les différentes occurrences du minimum. On pourrait
		l’écrire ainsi :
		
		\begin{LDA}
			\Algo{indicesMinimum}{\Par{tab}{\Array{n}{entiers}}}{Liste d'entiers}
				\Decl{min}{entier}
				\Decl{indicesMin}{Liste d'entiers}
				\Let min \Gets tab[0]
				\Let indicesMin \Gets \K{nouvelle} Liste d'entiers()
				\Stmt indicesMin.ajouter( 0 )
				\For{i}{1}{n-1}
					\If{tab[i] = min}
						\Stmt indicesMin.ajouter( i )
					\ElsIf{tab[i] < min}
						\Stmt indicesMin.vider() 
						\Stmt indicesMin.ajouter( i )
						\Let min \Gets tab[i]
					\EndIf \Comment rien à faire si tab[i] > min
				\EndFor
				\Return indicesMin
			\EndAlgo
		\end{LDA}
	
	%\clearpage
	%===================================
	\section{Comment implémenter l’état}
	%===================================
	
		Cette liste est bien utile mais comment la réaliser en pratique ?
		Comment représenter une liste variable d’éléments ? Pour
		l'instant, la seule structure qui peut accueillir
		plusieurs éléments de même type est le tableau. Nous allons donc
		prendre comme attribut principal de la liste, un tableau que nous
		appellerons \lda{éléments}. Comment, dès lors, contourner
		le problème de la limitation de la taille de ce tableau ?
		
		Repartons donc de la notion de tableau et tentons de comprendre sa
		limitation. Lors de sa création, un tableau se voit attribuer un espace
		bien précis et contigu en mémoire. Il se peut très bien que
		l'espace «~juste après~» soit occupé par une autre
		variable ce qui l'empêche de grandir. La parade est
		claire : si un \ tableau s’avère trop petit lors de son utilisation, il
		suffit d’en créer un autre plus grand ailleurs en mémoire et d’y
		recopier tous les éléments du premier. Évidemment, cette opération est
		coûteuse en temps et on cherchera à l’effectuer le moins souvent
		possible.
		
		\textbf{Quelle taille donner au nouveau tableau} ? L’idée qui vient
		immédiatement est d’augmenter la taille d’une unité afin d’accueillir
		le nouvel élément mais cette approche implique de fréquents
		agrandissements. Il est plus efficace d’augmenter la taille
		proportionnellement, par exemple en la multipliant par un facteur 2.
		
		\begin{center}
		\begin{tabular}{|m{0.259cm}|m{0.259cm}|m{0.259cm}|m{0.087999985cm}m{0.46000004cm}m{0.087999985cm}|m{0.25300002cm}|m{0.259cm}|m{0.259cm}|m{0.15299998cm}|m{0.15299998cm}|m{0.17cm}|}
		\hhline{---~~~------}
		 1 &
		 5 &
		 7 &
		~
		 &
		 ${\Rightarrow}$ &
		~
		 &
		 1 &
		 5 &
		 7 &
		 . &
		 . &
		 .\\\hhline{---~~~------}
		\end{tabular}
		\end{center}
		
		
		\textbf{Taille logique et taille physique}. À tout moment, le tableau
		aura une et une seule taille même si celle-ci pourra changer au cours
		du temps. Puisqu’on multipliera la taille du tableau par 2 pour des
		raisons d’efficacité, il y aura toutefois une différence entre la
		\textbf{taille physique} d’un tableau et sa \textbf{taille logique}. La
		taille physique est le nombre de cases réservées pour le tableau alors
		que la taille logique est le nombre de cases effectivement occupées.
		Dans ce qui suit, on s'arrangera pour que les cases
		occupées soient groupées à gauche du tableau (il n'y a
		pas de trou). Pour l’utilisateur, seule la taille logique a un sens (on
		lui cache les détails d’implémentation).
		
		\textbf{Exemple} : pour le tableau suivant, la taille logique est de 6
		(c’est cette taille qui a du sens pour l’utilisateur de la liste) et la
		taille physique est de 8.
		
		\begin{center}
		\begin{tabular}{|*{8}{>{\centering\arraybackslash}m{0.5cm}|}}
		\hline
		 2 &
		 5 &
		 4 &
		 8 &
		 3 &
		 12 &
		. &
		. \\\hline
		\end{tabular}
		\end{center}
		
		Quand il faut insérer un élément (en position valide) ou en ajouter un
		en fin de liste, deux cas se présentent :
		
		\begin{itemize}
			\item 
				si la taille logique est plus petite que la taille physique, il suffit
				d’ajouter l’élément dans le tableau et d’adapter la taille logique.
			\item 
				si la taille logique est égale à la taille physique, il faut
				procéder à un agrandissement du tableau.
		\end{itemize}
		
		\textbf{Les tableaux dynamiques}. 
		En \emph{DEV$_1$},
		nous n'avons vu que des tableaux
		qu'on appellera \emph{statiques},
		qui sont créés lors de leur déclaration.
		Ici, nous avons besoin de tableaux 
		qu'on appellera \emph{dynamiques},
		créés dans le code
		(comme le sont les tableaux en Java).
		
		Introduisons une notation.
		Un tableau dynamique sera déclaré puis créé ainsi :
		
			\begin{LDA}
				\Decl{tab}{\K{tableau} de chaines}
				\Let tab \Gets \K{nouveau} \Array{n}{chaines} \RComment n doit avoir une valeur
			\end{LDA}
		
		\textbf{Implémentation}.
		Présentons les attributs nécessaires et l'algorithme
		d’agrandissement du tableau.
		
		\begin{LDA}
			\Class{Liste de T}
				\Private
					\Decl{éléments}{\K{tableau} de T}
					\Decl{tailleLogique}{entier}
					\Decl{taillePhysique}{entier}
				\Private
					\Method {agrandir}{}{}
						\Decl{nouveauTab}{\K{tableau} de T}
						\Let taillePhysique \Gets taillePhysique * 2
						\Let nouveauTab \Gets \K{nouveau} \Array{taillePhysique}{T}
						\For{i}{0}{tailleLogique-1}
							\Let nouveauTab[ i ] \Gets éléments[ i ]
						\EndFor
						\Let éléments \Gets nouveauTab
					\EndMethod
				\EndClass
			\end{LDA}
		
		\bigskip
		
		\textbf{Réduction du tableau}. Tout comme on agrandit le tableau si
		nécessaire, on pourrait le réduire lorsque des suppressions d’éléments
		le rendent sous-utilisé (par exemple lorsque la taille logique devient
		inférieure au tiers de la taille physique). 
		Nous n'aborderons pas cette problématique cette année.
	
	
	%=======================================
	\section{Implémentation du comportement}
	%=======================================
	
		Nous avons à présent toutes les cartes en main pour écrire les méthodes
		publiques de la classe.
		
		\begin{LDA}
			\Constr{Liste de T}{}
				\Let tailleLogique \Gets 0
				\RComment la liste est vide au départ
				\Let taillePhysique \Gets 32
				\RComment une bonne valeur pour commencer
				\Let éléments \Gets \K{nouveau} \Array{taillePhysique}{T}
			\EndConstr
		\end{LDA}
		
		\begin{LDA}
			\Method{get}{pos : entier}{T}
				\If{ pos < 0 OU pos $\geq$ tailleLogique}
					\Error "position invalide"
				\EndIf
				\Return éléments[ pos ]
			\EndMethod
		\end{LDA}
		
		\begin{LDA}
			\Method{set}{pos : entier, valeur : T}{}
				\If{ pos < 0 OU pos $\geq$ tailleLogique}
					\Error "position invalide"
				\EndIf
				\Let éléments[ pos ] \Gets valeur
			\EndMethod
		\end{LDA}
		
		\begin{LDA}
			\Method{taille}{}{entier}
				\Return tailleLogique
				\RComment et pas la taille physique !
			\EndMethod
		\end{LDA}
		
		\begin{LDA}
			\Method{ajouter}{valeur : T}{}
				\If{tailleLogique = taillePhysique}
					\Stmt agrandir()
					\RComment méthode privée détaillée supra
				\EndIf
				\Let éléments[ tailleLogique ] \Gets valeur
				\Let tailleLogique \Gets tailleLogique + 1
			\EndMethod
		\end{LDA}
		
		\begin{LDA}
			\Method{insérer}{pos : entier, valeur : T}{}
				\If{ pos < 0 OU pos $\geq$ tailleLogique}
					\Error "position invalide"
				\EndIf
				\If{tailleLogique = taillePhysique}
					\Stmt agrandir()
				\EndIf
				\Stmt décalerDroite( pos )
				\RComment voir ci-dessous
				\Let tailleLogique \Gets tailleLogique + 1
				\Let éléments[ pos ] \Gets valeur
			\EndMethod
		\end{LDA}
		
		\begin{LDA}
			\Method{supprimer}{}{}
				\LComment supprime le dernier élément
				\If{tailleLogique = 0}
					\Error "liste vide"
				\EndIf
				\Let tailleLogique \Gets tailleLogique - 1
			\EndMethod
		\end{LDA}
		
		\begin{LDA}
			\Method{supprimerPos}{pos : entier}{}
				\If{ pos < 1 OU pos > tailleLogique}
					\Error "position invalide"
				\EndIf
				\Stmt décalerGauche( pos + 1 )
				\RComment voir méthode ci-dessous
				\Let tailleLogique\Gets tailleLogique - 1
			\EndMethod
		\end{LDA}
		
		\begin{LDA}
			\Method{supprimer}{valeur: T}{booléen}
				\Decl{estPrésent}{booléen}
				\Decl{pos}{entier}
				\Let estPrésent \Gets existe(valeur, pos)
				\If{estPrésent}
					\Stmt supprimer( pos )
				\EndIf
				\Return estPrésent
			\EndMethod
		\end{LDA}
		
		\begin{LDA}
			\Method{vider}{}{}
				\Let tailleLogique \Gets 0 
				\RComment Les éléments ne sont pas effacés mais sont ignorés
			\EndMethod
		\end{LDA}
		
		\begin{LDA}
			\Method{estVide}{}{booléen}
				\Return tailleLogique = 0
			\EndMethod
		\end{LDA}
		
		\begin{LDA}
			\Method{existe}{valeur\In : T, pos\Out : entier}{booléen}
				\Let pos \Gets 0
				\LComment Rq : le ET ci-dessous est une évaluation 
				court-circuitée (cf. le cours d'Algo en DEV1)
				\While{ pos {$<$} tailleLogique ET éléments[ pos ] {${\neq}$} valeur}
					\Let pos \Gets pos + 1
				\EndWhile
				\Return pos {$<$} tailleLogique
			\EndMethod
		\end{LDA}
		
		\begin{LDA}
			\LComment Ces méthodes-ci sont privées
			\bigskip
		
			\Method{décalerDroite}{début : entier}{}
				\LComment Décale tous les éléments d'une position vers
				la droite à partir de début
				\For[-1]{i}{tailleLogique-1}{début}
					\Let éléments[ i + 1 ] \Gets éléments[ i ]
				\EndFor
			\EndMethod
		
			\bigskip
			
			\Method {décalerGauche}{début : entier}{}
				\LComment Décale toutes les éléments d'une position vers
				la gauche à partir de début ; 
				\LComment ce paramètre vaut toujours au moins 2.
				\For{i}{début}{tailleLogique-1}
					\Let éléments[ i - 1 ] \Gets éléments[ i ]
				\EndFor
			\EndMethod
		\end{LDA}
		
		\bigskip
		
		\marginicon{attention}
		\textbf{La recherche se fait sur un élément complet.}
		
		Prenons comme exemple une liste de contacts.
		Lors d'une recherche, on doit fournir
		\textbf{tout} le contact à
		rechercher. Il s'agit juste de savoir
		s'il est présent et où. Une autre méthode intéressante
		serait de retrouver un contact à partir d'une partie
		de l'information, par exemple son nom. Cette méthode
		est fort proche de notre méthode de recherche mais il serait très
		difficile de l'écrire génériquement. On vous demandera
		d'écrire explicitement une telle méthode de recherche
		en cas de besoin.
	
	
	%====================================
	\section{Et sans tableau dynamique ?}
	%=====================================
	
		Certains langages (c’est le cas de Cobol) ne permettent pas de créer
		dynamiquement un nouveau tableau. Il vous faudra travailler avec un
		tableau classique en le créant suffisamment grand.
			
		Les algorithmes d’ajout/suppression/recherche vus pour la liste peuvent
		être appliqués tels quels à un tableau statique à une modification près
		: lors d’un ajout dans un tableau plein, on ne peut pas l’agrandir; il
		faut générer une erreur.
	
	%==================
	\section{Exercices}
	%===================
	
		\begin{Exercice}{Manipulation d'une liste}
			Écrire un module qui crée la liste suivante :
			\begin{enumerate}[start=0]
			\item 494
			\item 209
			\item 425
			\end{enumerate}
			affiche sa taille, demande si la valeur 425 est présente, 
			supprime la valeur 209 puis insére la valeur 101 en tête de liste.
		\end{Exercice}
		
		\begin{Exercice}{Liste des premiers entiers}
			Écrire un module qui reçoit un entier $n$ en paramètre et retourne la
			liste contenant les entiers de 1 à $n$ dans l'ordre
			décroissant. On peut supposer que $n$ est positif.
		\end{Exercice}
			
		\begin{Exercice}{Somme d'une liste}
			\marginicon{java}
			Écrire un module qui calcule la somme des éléments d’une liste
			d’entiers.
		\end{Exercice}
		
		\begin{Exercice}{Les extrêmes}
				\marginicon{java}
				Écrire un module qui supprime le minimum et le maximum des éléments
				d’une liste d’entiers. On peut supposer que le maximum et le minimum
				sont uniques.
		\end{Exercice}
		
		\begin{Exercice}{Anniversaires}
				%\marginicon{java}
				Écrire un module qui reçoit une liste de structure Personne 
				(nom + prénom + date de naissance) et retourne la liste
				de ceux qui sont nés durant un mois passé en paramètre 
				(donné sous la forme d'un entier entre 1 et 12).
		\end{Exercice}
			
		\begin{Exercice}{Concaténation de deux listes}
				\marginicon{java}
				Écrire un module qui reçoit 2 listes et ajoute
				à la suite de la première les éléments de la seconde; la seconde liste
				n'est pas modifiée par cette opération.
		\end{Exercice}
		
		\begin{Exercice}{Fusion de deux listes}
				\marginicon{java}
				Soit deux listes \textbf{ordonnées}
				d'entiers (redondances possibles). Écrire un module
				qui les fusionne. Le résultat est une liste encore ordonnée contenant
				tous les entiers des deux listes de départ (qu'on
				laisse inchangées).
		
				Exemple : Si les 2 listes sont (1, 3, 7, 7) et (3, 9), 
				le résultat est (1, 3, 3, 7, 7, 9).
		\end{Exercice}
		
		\begin{Exercice}{Le nettoyage}
			Écrire un module qui reçoit une liste de chaines en paramètre et
			supprime de cette liste tous les éléments de valeur donnée en
			paramètre. L'algorithme retournera le nombre de
			suppressions effectuées.
		\end{Exercice}
			
		\begin{Exercice}{Éliminer les doublons d'une liste}
				Soit une liste \textbf{ordonnée} 
				d'entiers avec de possibles redondances. Écrire un
				module qui enlève les redondances de la liste.
						
				Exemple : Si la liste est (1, 3, 3, 7, 8, 8, 8),
				le résultat est (1, 3, 7, 8).
		
				\begin{enumerate}[label=\alph*)]
					\item 
						Faites l'exercice en créant une \textbf{nouvelle
						liste} (la liste de départ reste inchangée)
					\item 
						Refaites l'exercice en \textbf{modifiant}
						la liste de départ (pas de nouvelle liste)
				\end{enumerate}
		\end{Exercice}
				
		\begin{Exercice}{Rendez-vous}
			Soit la structure «\lda{~RendezVous~}» composée d’une date
			et d’un motif de rencontre. Écrire un module qui reçoit une liste de
			rendez-vous et la met à jour en supprimant tous ceux qui sont désormais
			passés. 
		\end{Exercice}
		
	
