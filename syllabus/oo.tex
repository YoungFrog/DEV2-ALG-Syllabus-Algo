%========================
\chapter{L'orienté objet}
%=========================

	\marginicon{objectif}
	Le cours de Java vous a présenté la programmation orienté objet.
	Dans ce chapitre, nous allons rapidement revoir ce sujet
	et présenter comment nous allons l'utiliser dans ce cours.
	Nous nous contenterons de parler d'\textit{encapsulation}. 
	Les autres piliers de l'orienté objet 
	(\textit{héritage} et \textit{polymorphisme}) 
	ne seront pas vus cette année.

%===================
\section{Motivation}
%====================

	Au cours de Java,
	vous avez vu que l'orienté objet permet de structurer une
	application en regroupant dans un même \emph{objet}
	des données et le code qui va manipuler ces données.
	
	Une autre façon de voir l'orienté objet
	est de constater qu'une classe permet de définir 
	un nouveau \emph{type de données}.
	La notion de \emph{structure} permet déjà cela mais de façon limitée
	car elle ne reprend que des données et pas du code.
	Avec l'orienté objet, 
	on dispose de méthodes définissant ce qu'on peut faire avec des données
	(les objets) de ce type.
	C'est ainsi que nous l'utiliserons pour définir les listes 
	dans un prochain chapitre.

	
%=======================================
\section{Illustration~: une durée}
%=======================================

	Voyons tout cela au travers d'un exemple complet.
	Il est parfois utile d'avoir à sa disposition un type
	de données permettant de représenter une durée.	
	Utiliser plusieurs entiers (un pour les heures, un autre pour les minutes,
	un autre encore pour les secondes) n'est pas pratique.
	Utiliser une structure est déjà mieux mais offre moins d'avantage
	que l'orienté objet.	
	Voyons comment définir ce nouveau type de données en orienté objet.
	
	\subsection{Ce que l’on veut vraiment}
	
		Avant tout, il faut bien préciser ce que l’on veut décrire
		et bien faire la distinction entre un \emph{moment} et une \emph{durée}.
		L’«~heure~» est un concept multifacettes. 
		Parle-t-on de l’heure comme moment dans la journée 
		ou de l’heure comme représentant une durée ? 
		Dans le premier cas, elle ne peut dépasser 24h 
		et la différence entre 2 heures n’a pas de sens 
		(ou plus précisément n’est pas une heure, mais une durée !).
		Ce que nous nous proposons de créer ici est une durée,
		correspondand au deuxième cas.
		Et pour être plus précis encore,
		nous allons nous limiter à une précision à la seconde près,
		pas plus%
		\footnote{%
			Ajouter plus de précision ne serait pas plus compliqué à faire.%
		}.
	
	\subsection{Le comportement (les méthodes)}
	
		La première question à se poser est celle des services qu’on veut
		fournir, c’est-à-dire des méthodes publiques de la classe. On doit
		pouvoir \textit{construire} une durée. On doit pouvoir connaitre le
		nombre de jours, d’heures, minutes ou secondes correspondant à une durée. On doit
		pouvoir effectuer des calculs avec des durées (addition, soustraction).
		Enfin, on doit pouvoir comparer des durées. Arrêtons-nous là, mais en
		pratique, on pourrait trouver encore bon nombre d’autres méthodes qu’il
		serait intéressant de fournir. 
		
		Voici comment nous allons noter tout cela au cours d'algorithmique.
		
		\begin{LDA}
			\Class{Durée}
				%\Private
					%\LComment rien encore
				\Public
					\ConstrSign{Durée}{secondes~: entier}
					\ConstrSign{Durée}{heure, minute, seconde~: entiers}
					\Empty
					\MethodSign{getJour}{}{entier}
					\RComment nb de jours dans une durée
					\MethodSign{getHeure}{}{entier}
					\RComment entier entre 0 et 23 inclus
					\MethodSign{getMinute}{}{entier}
					\RComment entier entre 0 et 59 inclus
					\MethodSign{getSeconde}{}{entier}
					\RComment entier entre 0 et 59 inclus
					\Empty
					\MethodSign{getTotalJours}{}{entier}
					\RComment Le nombre total de jours
					\MethodSign{getTotalHeures}{}{entier}
					\RComment Le nombre total d’heures
					\MethodSign{getTotalMinutes}{}{entier}
					\RComment Le nombre total de minutes
					\MethodSign{getTotalSecondes}{}{entier}
					\RComment Le nombre total de secondes
					\Empty
					\MethodSign{ajouter}{autreDurée~: Durée}{}
					\MethodSign{différence}{autreDurée~: Durée}{Durée}
					%\MethodSign{égale}{autreDurée~: Durée}{booléen}
					\MethodSign{plusPetit}{autreDurée~: Durée}{booléen}
			\EndClass
		\end{LDA}
		
		\marginicon{attention}
		\textbf{Quelques remarques}
		\begin{itemize}
			\item
				On a deux constructeurs, ce qui offre plus de souplesse pour initialiser
				un objet. Ceci est un exemple supplémentaire du concept de
				«\textbf{~surcharge~}».
			\item
				Faisons bien la distinction entre les méthodes
				\lda{getXXX()} et \lda{getTotalXXX()}. Par
				exemple, la méthode \lda{getMinute()} retourne la valeur
				de la composante «~minutes~» dans une représentation HMS tandis que la
				méthode \lda{getTotalMinutes()} retourne le nombre total
				de minutes entières pour cette durée. Ex~: pour 1h23’12’’,
				\lda{getMinute()} retourne 23 et
				\lda{getTotalMinutes()} retourne 83. Idem avec 
				les jours, les heures et les secondes.
			\item 
				Les méthodes \lda{getTotalXXX()} retournent le nombre
				(toujours entier) de XXX contenus dans la durée. Exemple, avec la durée
				0h23’52'’, \lda{getTotalMinutes()}
				retourne 23 et pas 24 (autrement dit, il n’y a pas d’arrondi vers le
				haut).
			\item 
				Il n’y a pas de \textit{mutateur }(\lda{setXXX()}). Ce qui
				signifie qu’on ne peut pas changer directement la valeur de l’objet
				après son initialisation. On aurait pu en définir mais nous
				n'avons pas jugé utile de le faire dans ce cas
				précis.
			\item 
				La méthode \lda{ajouter()} ne retourne rien. En effet,
				elle ajoute la durée à l’objet sur lequel est appelée la méthode. C’est
				un choix ; on aurait aussi pu dire que la méthode ne modifie pas
				l’objet mais en retourne un autre qui représente la somme. Dans ce cas,
				on l’aurait plutôt appelée «\lda{~plus( )}~».
			\item 
				La méthode \lda{différence()}, elle, renvoie toujours une
				durée (positive).
			\item 
				Nous ne définissons pas de méthode d'affichage
				similaire au \lda{toString()} qu'on retrouve en Java.
				L'affichage correct de l'information ne fait pas partie
				des préoccupations de ce cours.
				On supposera que "\lda{\K{afficher} objet}"
				affiche correctement les données associées à l'objet.
		\end{itemize}
	
	\subsection{La représentation de l'état (les attributs)}
	
		La question suivante est~: «~Comment représenter une durée en interne ?
		». Plusieurs possibilités existent. Par exemple~:	
		\begin{itemize}
			\item 
				via le nombre d’heures, de minutes et de secondes
			\item 
				via le nombre total de secondes
			\item 
				via une chaine, par exemple au format «~HH~:MM~:SS~» où HH pourrait
				éventuellement excéder 23.
		\end{itemize}
		
		Le premier choix semble le plus évident mais réfléchissons-y de plus
		près. D’une part, pourquoi se limiter aux heures. On pourrait
		introduire un champ ‘\lda{jour}’ (après tout on a bien
		une méthode \lda{getJour()}). 
		
		Quel critère doit vraiment nous permettre de décider ? Il faut une
		représentation qui soit suffisante (tout est représenté) et qui
		permette d’écrire des méthodes lisibles et si possible efficaces
		(c'est-à-dire où le calcul est rapide). Selon ces
		critères, la deuxième représentation est de loin la meilleure. 
		
		Voilà comment nous indiquons les attributs d'une classe.
		
		\begin{LDA}
			\Class{Durée}
				\Private
					\Decl{totalSecondes}{entier}
				\Public
					\LComment idem
			\EndClass
		\end{LDA}
	
	\subsection{L'implémentation}
	
		On est à présent prêt pour écrire le code des méthodes. 
		Pour une meilleure lisibilité,
		nous gardons les signatures des méthodes dans la classe
		et nous détaillons leur contenu en dehors.
		Ce qui donne :
		
		\begin{LDA}
			\Class{Durée}
				\Private
					\Decl{totalSecondes}{entier}
				\Public
					\ConstrSign{Durée}{secondes~: entier}
					\ConstrSign{Durée}{heure, minute, seconde~: entiers}
					\Empty
					\MethodSign{getJour}{}{entier}
					\RComment nb de jours dans une durée
					\MethodSign{getHeure}{}{entier}
					\RComment entier entre 0 et 23 inclus
					\MethodSign{getMinute}{}{entier}
					\RComment entier entre 0 et 59 inclus
					\MethodSign{getSeconde}{}{entier}
					\RComment entier entre 0 et 59 inclus
					\Empty
					\MethodSign{getTotalHeures}{}{entier}
					\RComment Le nombre total d’heures
					\MethodSign{getTotalMinutes}{}{entier}
					\RComment Le nombre total de minutes
					\MethodSign{getTotalSecondes}{}{entier}
					\RComment Le nombre total de secondes
					\Empty
					\MethodSign{ajouter}{autreDurée~: Durée}{}
					\MethodSign{différence}{autreDurée~: Durée}{Durée}
					%\MethodSign{égale}{autreDurée~: Durée}{booléen}
					\MethodSign{plusPetit}{autreDurée~: Durée}{booléen}
			\EndClass
		\end{LDA}

		\begin{LDA}
			\Constr{Durée}{secondes~: entier}
				\If{secondes < 0}
					\Stmt \K{erreur} "paramètre négatif"
				\EndIf
				\Let totalSecondes \Gets secondes
			\EndConstr
		\Empty
			\Constr{Durée}{heure, minute, seconde~: entiers}
				\If{heure < 0 OU minute < 0 OU seconde < 0}
					\Stmt \K{erreur} "un des paramètres est négatif"
				\EndIf
				\Let totalSecondes \Gets 3600*heure + 60*minute + seconde
			\EndConstr
		\Empty
			\LComment Retourne le nombre de jours dans une 
			représentation JJ/HH:MM:SS
			\Method{getJour}{}{entier}
				\Return totalSecondes DIV (3600*24)
			\EndMethod
		\Empty
			\LComment Retourne le nombre d'heures dans une 
			représentation JJ/HH:MM:SS
			\Method{getHeure}{}{entier}
				\LComment On doit enlever les jours éventuels
				\Return (totalSecondes DIV 3600) MOD 24
			\EndMethod
		\Empty
			\LComment Retourne le nombre de minutes dans une 
			représentation JJ/HH:MM:SS
			\Method{getMinute}{}{entier}
				\LComment On doit enlever les heures éventuelles
				\Return (totalSecondes DIV 60) MOD 60
			\EndMethod
		\Empty
			\LComment Retourne le nombre de secondes dans une 
			représentation JJ/HH:MM:SS
			\Method{getSeconde}{}{entier}
				\LComment On doit enlever les minutes éventuelles
				\Return totalSecondes  MOD 60
			\EndMethod
		\Empty
			\LComment Retourne le nombre entier d’heures complètes
			\Method{getTotalHeures}{}{entier}
				\Return totalSecondes DIV 3600
			\EndMethod
		\Empty
			\LComment Retourne le nombre entier de minutes complètes
			\Method{getTotalMinutes}{}{entier}
				\Return totalSecondes DIV 60
			\EndMethod
		\Empty
			\LComment Retourne le nombre entier de secondes complètes
			\Method{getTotalSecondes}{}{entier}
				\Return totalSecondes
			\EndMethod
		\Empty
			\Method{ajouter}{autreDurée~: Durée}{}
				\Let totalSecondes \Gets totalSecondes + autreDurée.totalSecondes
			\EndMethod
		\Empty
			\Method{différence}{autreDurée~: Durée}{Durée}
				\Return \K{nouvelle} Durée(valeurAbsolue(totalSecondes - autreDurée.totalSecondes))
			\EndMethod
		%\Empty
			%\Method{égale}{autreDurée~: Durée}{booléen}
				%\Return totalSecondes = autreDurée.totalSecondes
			%\EndMethod
		\Empty
			\Method{plusPetit}{autreDurée~: Durée}{booléen}
				\Return totalSecondes < autreDurée.totalSecondes
			\EndMethod
		\end{LDA}

%=====================================
\section{Utiliser}
%=====================================

	Pour utiliser le nouveau type de donnée créé,
	il faut l'instancier, c'est-à-dire créer un nouvel objet de ce type.
	Nous allons reprendre une notation très proche de Java.
	
	Illustrons cela au travers d'un petit algorithme
	qui calcule la différence entre deux durées.
	
	\begin{LDA}
		\Algo{diffDurée}{}{}
			\Decl{durée1, durée2}{Durée}	\RComment Les variables sont déclarées/créées
			\Let durée1 \Gets \K{nouveau} Durée(3, 4, 49)	\RComment Les objets sont créés
			\Let durée2 \Gets \K{nouveau} Durée(3, 24, 37)	\RComment Les objets sont créés
			\Write durée2.différence(durée1)
		\EndAlgo
	\end{LDA}
	
%=====================================
\section{Quelques éléments de syntaxe}
%=====================================

	Clarifions certaines notations liées aux objets.

	\begin{itemize}
		\item
			On peut directement afficher un objet. 
			Cela affiche son état, c'est-à-dire les valeurs de ses attributs.

			\begin{LDA}
				\Decl{rendezVous}{Durée}
				\Let rendezVous \Gets \K{nouveau} Durée(14, 23, 56)
				\Write rendezVous 
				\RComment affichera 14, 23 et 56 dans un format lisible quelconque
			\end{LDA}
		\item
			De même, on peut directement lire un objet,
			ce qui a pour effet de créer un objet avec un état
			correspondant aux valeurs lues pour ses attributs.

			\begin{LDA}
				\Decl {rendezVous}{Durée}
				\Read rendezVous
			\end{LDA}
		\item 
			Le signe «~=~» peut être utilisé pour comparer deux objets.
			Ils seront considérés comme égaux s'ils sont dans le même état, 
			c'est-à-dire que leurs attributs ont la même valeur.
		\item
			Lorsqu'on déclare un objet, il n'est pas encore créé.
			On peut utiliser la valeur spéciale «~rien~»
			pour indiquer ou tester qu'un objet n'est pas encore créé.
			
			\begin{LDA}
				\Decl{parcours}{Durée}								\RComment parcours = rien
				\Let parcours \Gets \K{nouveau} Durée( 14, 23, 56 )	\RComment parcours ${\neq}$ rien
				\If{parcours $\neq$ rien}
					\Let parcours \Gets rien						\RComment parcours = rien
				\EndIf
			\end{LDA}
	\end{itemize}
			
%%==========================
%\section{La notion d'objet}
%%==========================

	%\subsection{Définition}
	%%-----------------------
	
		%\marginicon{definition}
		%Un \textbf{objet}%
		%\footnote{%
			%Les définitions sont tirées du livre de Cardon et
			%Dabancourt (cf. bibliographie)
		%}
		%est une entité logicielle qui~:
	
		%\begin{liste}
		%\item 
			%a une \textbf{identité~}; c'est-à-dire que nous pouvons
			%identifier un objet par un nom (tout comme une variable possède un
			%nom).
		%\item 
			%est capable de sauvegarder un \textbf{état}, c'est-à-dire 
			%un ensemble d'informations dans des variables
			%internes;
		%\item 
			%répond à des \textbf{messages} précis en déclenchant des activations
			%internes appropriées qui peuvent changer l'état de
			%l'objet. Ces opérations sont appelées des
			%\textbf{méthodes}. Ce sont des fonctions liées à des objets et qui
			%précisent le \textbf{comportement} de ces objets.
		%\end{liste}

	%\subsection{État}
	%%-----------------

		%\marginicon{definition}
		%Un objet contient de l'information, des données qui
		%définissent son état.

		%\textbf{Exemples}	
		%\begin{liste}
		%\item 
			%Pour un produit, l'état peut être~:
			%l'intitulé du produit, son code barre, son prix\dots 
		%\item 
			%Pour un employé, on peut avoir~: son nom, son prénom, son adresse, sa
			%date d'embauche, son salaire mensuel, sa fonction, son
			%téléphone\dots
		%\item 
			%Une carte à jouer a une couleur et une valeur.
		%\item 
			%L'état d'une date est le jour du
			%calendrier qu'elle représente.
		%\item 
			%L'état d'une heure est le moment de la
			%journée qu'elle représente.
		%\end{liste}

		%L'état d'un objet est mémorisé via des
		%variables qu'on appelle des \textit{attributs}.

	%\subsection{Attributs}
	%%----------------------

		%\marginicon{definition}
		%Les \textbf{attributs} d'un objet sont
		%l'ensemble des informations se présentant sous forme
		%de variables et permettant de représenter l'état
		%d'un objet.

		%Nous verrons plus loin la syntaxe précise 
		%pour définir les attributs d'un objet.

		%\textbf{Exemples}
		%\begin{liste}
		%\item 
			%L'intitulé d'un produit peut être
			%représenté par une chaine. C'est également le cas des
			%nom(s) et prénom(s) d'un employé.
		%\item 
			%La date d'embauche peut être représentée par un «~objet
			%date~» (une date est rarement un type primitif du langage utilisé). Un
			%attribut d'un objet peut être lui même un objet.
		%\item 
			%Un moment de la journée peut aussi être un objet représenté par trois
			%entiers\footnote{Toutefois, on verra que ce n'est
			%peut-être pas la meilleure solution.}~: les heures, les minutes et les
			%secondes (en supposant qu'on désire une précision de
			%l'ordre de la seconde).
		%\item 
			%L'adresse d'un employé peut être
			%représentée par une seule chaine mais également par un «~objet
			%adresse~» (qui contiendrait~: une rue, un numéro, un code postal\dots).
		%\end{liste}

		%\marginicon{attention}
		%\textbf{Remarque}\textbf{~: }Certaines parties de
		%l'état peuvent évoluer au fil du temps.
		%D'autres parties sont immuables. 
		%Ainsi l'adresse d'une personne peut changer
		%mais pas sa date de naissance.

		%\begin{Emphase}{Exercices - attributs}
			%\marginicon{exercice}
			%\vskip-\baselineskip
			%\begin{enumerate}
				%\item 
					%Quel(s) attribut(s) prendriez-vous pour représenter
					%(l'état d') une date ?
				%\item 
					%Et pour un dé à 6 faces ?
				%\item 
					%Et pour un produit de magasin ?
				%\item 
					%Et pour une télévision ?
					%(on peut en trouver vraiment beaucoup !)
			%\end{enumerate}
		%\end{Emphase}

	%\subsection{Comportement}
	%%-------------------------

		%\marginicon{definition}
		%Le \textbf{comportement} d'un objet est défini par
		%l'ensemble des messages ou requêtes auxquels il peut répondre.

		%Pour ce faire, il exécute un module qui pourra
		%éventuellement retourner une information à l'émetteur
		%du message.
		
		%Les messages peuvent interroger l'objet, le modifier,
		%lui demander d'agir sur son environnement (afficher du
		%texte, modifier un fichier\dots). 

		%\textbf{Exemples}
		%\begin{liste}
		%\item
			%Quels «~messages~» peut-on envoyer à une date ? 
			%On peut lui demander (entre autres) :
			%\begin{liste}
			%\item
				%des informations sur le jour du mois, le mois, l'année,
				%le jour de la semaine ;
			%\item
				%si elle est antérieure ou non à une autre date ;
			%\item
				%si elle fait partie d'une année bissextile ;
			%\item
				%le nombre de jours qui la sépare de la fin de l'année ;
			%\item 
				%de passer au jour suivant, à la semaine suivante\dots
			%\end{liste}
		%\item 
			%Et pour un stock de produits ? On peut 
			%\begin{liste}
			%\item
				%lui demander la quantité disponible d'un produit donné ;
			%\item
				%lui annoncer l'arrivée d'une quantité
				%donnée d'un produit donné ;
			%\item
				%lui indiquer qu'un produit n'existe
				%plus (à retirer du stock) ;
			%\item 
				%lui demander d'enlever une certaine quantité
				%d'un produit du stock.
			%\end{liste}
		%\item
			%Et pour un employé ? On peut
			%\begin{liste}
			%\item 
				%lui demander son adresse, son salaire ou sa fonction\dots
			%\item 
				%augmenter son salaire ;
			%\item 
				%le changer de fonction ;
			%\item 
				%le licencier 
				%(penser à prévoir une date de départ dans l'état !).
			%\end{liste}
		%\item 
			%Pour un moment de la journée on peut demander s'il se
			%situe le matin ou pas\dots	
		%\end{liste}

		%\begin{Emphase}{Exercices - comportement}
			%\marginicon{exercice}
			%\vskip-\baselineskip
			%\begin{enumerate}
			%\item
				%Quel comportement voyez-vous pour un téléviseur ?
			%\item
				%Et pour un produit de magasin ?
			%\end{enumerate}
		%\end{Emphase}

	%\subsection{Méthode}
	%%--------------------
	
		%\marginicon{definition}
		%Un message lance l'exécution d'un
		%module appelé \textbf{méthode} dans le jargon de
		%l'orienté objet.

		%\textbf{Exemples}
		%\begin{liste}
		%\item
			%Pour permettre à une date de passer au jour suivant, nous allons définir
			%une méthode qui incrémente le jour du mois en tenant compte
			%d'un possible basculement au mois suivant ou à
			%l'année suivante.
		%\item
			%Pour calculer le bénéfice d'un produit, nous allons définir
			%une méthode qui, à partir du prix d'achat et du prix de vente,
			%calcule le bénéfice.
		%\item 
			%Pour permettre à un moment d'indiquer
			%s'il est le matin ou pas, nous allons définir une
			%méthode comme celle-ci (nous verrons plus tard comment
			%l'associer aux objets)
	
			%\medskip
			%\begin{Pseudocode}
				%\LComment On suppose que 'heure' est un des attributs utilisés
				%\LComment pour représenter l'état (le moment dans la journée)
				%\Method{estMatin}{}{booléen}
					%\Return heure < 12 \RComment on considère que midi est situé l'après-midi
				%\EndMethod
			%\end{Pseudocode}
			
			%\medskip
			%Cet exemple devrait vous sembler familier à deux exceptions près
	
			%\begin{liste}
				%\item
					%on utilise le mot «~\pseudocode{méthode}~» en lieu et place de
					%«~\pseudocode{module}~» ;
				%\item
					%les attributs (l'heure ici) ne sont pas passés en paramètre.
					%Une méthode appartient à un objet et connait
					%les attributs de cet objet. 
					%Nous verrons plus loin la syntaxe précise.
			%\end{liste}
		%\end{liste}

		%\begin{Emphase}{Exercices - méthodes}
			%\vskip-\baselineskip
			%\marginicon{exercice}
			%\begin{enumerate}
			%\item 
				%Dans le comportement d'un téléviseur, on retrouve
				%«~éteindre~» et «~allumer~». 
				%À quoi ressemblerait le code de ces méthodes ?
			%\item 
				%Écrivez la méthode qui permet de passer au jour suivant.
			%\item 
				%Écrivez la méthode qui calcule 
				%le bénéfice réalisé lors de la vente d'un produit.
			%\end{enumerate}
		%\end{Emphase}

	%\subsection{Activer un comportement}

		%Pour activer un comportement d'un objet, 
		%il faut lui envoyer un message 
		%(ou dit autrement, appeler une de ses méthodes). 
		%La syntaxe que nous allons utiliser 
		%(c'est la plus courante) est la notation pointée.

		%\begin{Pseudocode}
			%\Stmt nomObjet.nomMéthode()
		%\end{Pseudocode}

		%\medskip\textbf{Exemple}~:
		%Supposons que le nom «~maintenant~» 
		%désigne un objet contenant un moment de la journée 
		%(on verra comment réaliser cela). 
		%Si on veut savoir si on est le matin, on peut écrire

		%\begin{Pseudocode}
			%\If{maintenant.estMatin()}
				%\Stmt ...
			%\EndIf
		%\end{Pseudocode}

		%\begin{Emphase}{Exercice – activer un comportement}
			%\marginicon{exercice}
			%Écrire la portion de code qui allume une télévision 
			%(désignée par «~maTélévision~») 
			%et puis l'éteint aussitôt après.		
		%\end{Emphase}

	%\subsection{Les paramètres d'un comportement}

		%Activer un comportement revient à appeler une méthode de
		%l'objet. Souvent il est nécessaire
		%d'envoyer à l'objet des informations
		%complémentaires pour préciser notre demande ce qui se fait via
		%l'utilisation des paramètres.

		%\textbf{Exemple}~:
		%Si on veut modifier le salaire d'un employé, 
		%il faut que notre message contienne le nouveau salaire. 
		%Autrement dit, 
		%il faut communiquer ce nouveau salaire à la méthode 
		%de changement du salaire.
		%Ce qui donne la méthode suivante~:

		%\begin{Pseudocode}
			%\Method{modifierSalaire}{nouveauSalaire~: entier}{}
				%\Stmt salaire \Gets nouveauSalaire
			%\EndMethod
		%\end{Pseudocode}

		%\begin{Emphase}{Exercices – paramètres du comportement}
			%\vskip-\baselineskip
			%\marginicon{exercice}
			%\begin{enumerate}
			%\item 
				%Prenons un objet représentant un produit de magasin. 
				%Nous supposerons qu'un produit a un \textit{numéro}, 
				%un  \textit{libellé}, un \textit{prixAchat}, 
				%un \textit{prix de vente} et une \textit{quantitéEnStock}.
				%Donnez les \textbf{entêtes} des méthodes suivantes 
				%qui permettent de~:
				%\begin{liste}
				%\item 
					%obtenir le prix de vente
				%\item 
					%calculer le bénéfice
				%\item 
					%donner la quantité restant en stock
				%\item 
					%dire si le produit est en rupture de stock.
				%\end{liste}
						
			%\item 
				%Prenons un objet représentant une date du calendrier grégorien. 
				%Donnez les entêtes des méthodes suivantes qui permettent de~:
				%\begin{liste}
				%\item
					%demander le nom du jour correspondant (par exemple "lundi", "mardi"\dots)
				%\item 
					%savoir si une date est antérieure à une autre
				%\item 
					%connaitre le nombre de jours (absolu) séparant deux dates.
				%\end{liste}
					
			%\item 
				%Utilisation. Soit deux dates $date1$ et $date2$ ; écrivez la
				%portion de code qui utilise les méthodes ci-dessus pour
				%\begin{liste}
					%\item 
						%vérifier quelle date précède l'autre;
					%\item 
						%calculer le nombre de jours d'écart entre ces deux
						%dates.
				%\end{liste}
			%\end{enumerate}
		%\end{Emphase}
	
%%=========================
%\section{L'encapsulation}
%%=========================

	%Un objet possède un état qui est représenté par des attributs. 
	%Les bonnes pratiques de la programmation orientée objet préconisent
	%fortement que les attributs d'un objet soient
	%invisibles en dehors de l'objet. 
	%Ils ne pourront être accédés qu'au travers 
	%du comportement de l'objet, 
	%c'est-à-dire via ses méthodes.
	
	%\clearpage
	%\marginicon{definition}
	%\textbf{%
		%Lorsque les détails de l'implémentation
		%d'un objet sont masqués aux autres objets, on dit
		%qu'il y a \textbf{encapsulation} des données et du
		%comportement des objets.
	%}
	
	%Pourquoi une telle recommandation ? 
	%Le but est de garantir la cohérence de l'état de l'objet. 
	%Si on pouvait accéder directement à un attribut 
	%(et donc le modifier), 
	%on pourrait y mettre une valeur incohérente. 
	%Par exemple, on pourrait dire que les minutes d'un moment 
	%valent -3 ou 75 ou encore que le jour d'une date est 32 !
	
	%Dès lors, il nous faudra préciser pour chaque \textbf{membre} 
	%(attributs et méthodes) d'un objet s'il est
	%\textbf{privé} (inconnu de l'extérieur) ou
	%\textbf{public} (connu de l'extérieur). 
	
	%Le bon usage impose que tous les attributs soient rendus privés 
	%et que les méthodes restent publiques. 
	%Toutefois, on pourra trouver également des méthodes privées. 
	%Ce sera notamment le cas si plusieurs méthodes d'un objet 
	%ont une partie commune ; 
	%il sera intéressant de la \textit{factoriser}, 
	%c-à-d en faire une méthode privée (ex~: un calcul de maximum).
	
	%Puisqu'un attribut est privé,
	%il est courant pour chacun des attributs de rencontrer 
	%une méthode destinée à connaitre la valeur de cet attribut 
	%et une autre qui permet de la modifier.

	%\subsection{Accesseur et mutateur}
	%%----------------------------------
	
	%\marginicon{definition}
	
	%\textbf{Accesseur}%
	%\footnote{On utilise aussi souvent le mot anglais «~getter~».}%
	%~: méthode dont le but est de fournir la valeur d'un attribut.

	%\textbf{Mutateur}%
	%\footnote{On utilise aussi souvent le mot anglais «~setter~»}%
	%~: méthode dont le but est de modifier la valeur d'un attribut.

	%Par convention, 
	%ces méthodes sont nommées \pseudocode{getNom} et 
	%\pseudocode{setNom} où «~nom~» est le nom de l'attribut%
	%\footnote{%
		%Pour un attribut booléen, 
		%on pourra préférer \pseudocode{estNom} ou \pseudocode{isNom} 
		%au lieu de \pseudocode{getNom}. 
	%}
	%Par facilité, on utilisera parfois le terme «~accesseur~»
	%pour désigner à la fois les «~accesseurs~» et les «~mutateurs~».

	%\textbf{Exemple}~:
	%Écrivons l'accesseur et le mutateur pour l'attribut 
	%«~heure~» d'un moment de la journée.

	%\begin{Pseudocode}
		%\Method{getHeure}{}{entier}
			%\Return heure
		%\EndMethod
	%\end{Pseudocode}

	%\begin{Pseudocode}
		%\Method{setHeure}{uneHeure~: entier}{}
			%\Stmt heure \Gets uneHeure
		%\EndMethod
	%\end{Pseudocode}

	%\subsection{Que faire si le paramètre est invalide ?}
	%%-----------------------------------------------------
	
	%Dans l'exemple précédent, 
	%que se passerait-il si le paramètre \pseudocode{uneHeure} vaut 25 ? 
	%Une valeur aberrante serait affectée à l'attribut \pseudocode{heure}.

	%Dans le cas de paramètres invalides, 
	%la plus mauvaise solution est de ne rien faire. 
	%Le programme continuerait en croyant que tout s’est bien
	%passé et il court à la catastrophe. 
	%Il est préférable qu’un programme s'interrompe 
	%plutôt que de fournir une mauvaise réponse. 

	%Dans certains langages (comme le C), 
	%l’usage est que chaque module retourne un entier indiquant 
	%s'il y a eu une erreur (et laquelle). 
	%L’inconvénient est que le module appelant n’est pas
	%obligé de tenir compte de l’erreur.

	%Les \textbf{exceptions} sont un mécanisme du même genre 
	%mais qui oblige à fournir un code de traitement de l’erreur. 
	%Il ne sera pas étudié en première année%
	%\footnote{%
		%En tout cas pas au cours d'algorithmique 
		%mais vous étudierez cette notion au cours de Java.%
	%}.

	%Cette année, nous nous contenterons d'indiquer
	%clairement dans nos codes qu'il s'agit d'une situation anormale 
	%via la primitive \pseudocode{erreur} qui arrête le déroulement du
	%programme avec une courte explication du problème.

	%La syntaxe que nous allons retenir est

	%\begin{Pseudocode}
		%\Stmt \K{erreur} "explication de l'erreur"
	%\end{Pseudocode}

	%Ce qui donne~:	
	
	%\begin{Pseudocode}
		%\Method{setHeure}{uneHeure~: entier}{}
			%\If{uneHeure < 0 OU uneHeure > 23}
				%\Stmt \K{erreur} "heure invalide"
			%\EndIf
			%\Stmt heure \Gets uneHeure
		%\EndMethod
	%\end{Pseudocode}

	%%\medskip
	%%\textbf{Remarque}~: 
	%%Accéder à la valeur d’un attribut ne pose pas 
	%%de problème de validité. 
	%%Alors pourquoi rendre l’attribut complètement privé 
	%%plutôt que de n’empêcher que des modifications directes 
	%%(une sorte d’attribut en \textit{lecture seule}) ? 
	%%Cela permet de changer de façon transparente la représentation des données. 
	%%Nous y reviendrons plus loin.

	%%\begin{Emphase}{Exercice - encapsulation}
		%%\marginicon{exercice}
		%%Sans le savoir, vous avez déjà défini des accesseurs et des 
		%%mutateurs pour le téléviseur. 
		%%Lesquels ? 
		%%En suivant la convention de nom pour les accesseurs et les mutateurs, 
		%%quels noms auraient-ils dû porter ?
	%%\end{Emphase}

%%====================================================
%\section{La notion de classe et d'instance}
%%=====================================================

	%Pour pouvoir utiliser des objets nous allons devoir les définir
	%(expliciter leur état et leur comportement). 
	%Cette définition est commune à tous les objets similaires. 
	%Par exemple tous les moments ont un même comportement 
	%et un même type d'état 
	%(des heures, des minutes et des secondes).
	
	%\marginicon{definition}
	%\textbf{Une \textbf{classe} est un ensemble d'objets qui ont en
	%commun les mêmes méthodes et qui partagent les mêmes types
	%d'attributs.}

	%Une \textbf{instance}%
	%\footnote{%
		%Vous pouvez considérer les termes
		%«~instance de classe~» et «~objet~» comme synonymes.
	%}
	%d'une classe est un objet particulier
	%d'une classe qui peut activer les méthodes de la
	%classe et qui a des valeurs particulières pour ses attributs.

	%On peut établir le parallélisme avec les types de base que vous avez
	%déjà vus. Définir une classe revient à définir un nouveau type de
	%données. En gros, on peut dire qu'un \textbf{objet
	%est à une classe ce qu'une variable est à un type}.
	
	%Comprenons bien que les objets d'une même classe ont le
	%même «~type~» d'état mais pas le même état proprement
	%dit. Deux objets «~moment~» représentent tous deux un moment 
	%(heures, minutes, secondes) de la journée mais pas (forcement) 
	%le même ! Ils auront donc les mêmes attributs mais
	%avec des valeurs différentes !
	
	%\subsection{Définition d'une classe}
	%%------------------------------------
	
		%Nous devons d'abord définir une classe avant de pouvoir
		%en instancier les objets que nous voulons utiliser. Précisons la
		%syntaxe utilisée pour définir une classe

		%\begin{Pseudocode}
			%\Class{NomDeLaClasse}
				%\Private
					%\LComment liste des attributs (donc privés par convention)
				%\Public
					%\LComment liste des méthodes publiques
				%\Private
					%\LComment liste des méthodes privées
			%\EndClass
			%\LComment Par souci de lisibilité, on pourra indiquer uniquement 
			%les entêtes des
			%\LComment méthodes et donner le code complet des méthodes à 
			%la suite de la classe.
		%\end{Pseudocode}

		%\textbf{Exemple}~: 
		%la classe Moment qui représente un moment de la journée.

		%\begin{Pseudocode}
			%\Class{Moment}
				%\Private
					%\Decl heure~: entier
					%\Decl minute~: entier
					%\Decl seconde~: entier
				%\Public
					%\MethodSign{getHeure}{}{entier}
					%\MethodSign{getMinute}{}{entier}
					%\MethodSign{getSeconde}{}{entier}
					%\MethodSign{setHeure}{uneHeure~: entier}{}
					%\MethodSign{setMinute}{uneMinute~: entier}{}
					%\MethodSign{setSeconde}{uneSeconde~: entier}{}
					%\MethodSign{estMatin}{}{booléen}
			%\EndClass
			%\Empty
			%\Method{estMatin}{}{booléen}
				%\Return heure < 12
			%\EndMethod
			%\LComment + les accesseurs et les mutateurs		
		%\end{Pseudocode}

	%\subsection{Instanciation d'une classe}
	%%---------------------------------------
	
		%«~Instancier~» signifie créer un objet d'une classe.
		%Cela s'écrit avec l'instruction
		%\pseudocode{nouveau}. 
		%Pour lui donner un nom, 
		%on l'assigne à une variable déclarée du type de la
		%classe.

		%\begin{Pseudocode}
			%\Decl nomObjet~: nomClasse 
			%\RComment déclaration de l'objet
			%\Let nomObjet \Gets \K{nouveau} nomClasse() 
			%\RComment instanciation de l'objet
		%\end{Pseudocode}

		%Dans ce cours de Logique, nous adopterons le fait que 
		%les noms des paramètres soient différents de ceux des 
		%attributs (on préconisera d’imaginer des noms variés tels 
		%que uneDate, maDate, laListe, autreObjet\dots), ce qui évitera 
		%toute ambiguïté (entre minuscule et majuscule par exemple).

		%\textbf{Exemple}~: pour créer un moment de la journée.
		
		%\begin{Pseudocode}
			%\Module{test}{}{}
				%\Decl midi~: Moment
				%\RComment déclaration
				%\Let midi \Gets \K{nouveau} Moment()
				%\RComment instanciation
				%\Stmt midi.setHeure( 12 )
				%\RComment mutateur
				%\Stmt midi.setMinute( 0 )
				%\RComment " "
				%\Stmt midi.setSeconde( 0 )
				%\RComment " "
				%\If{midi.estMatin()}
					%\Write "Midi est considéré comme
					%étant encore le matin"
				%\Else 
					%\Write "Midi est considéré comme
					%étant l'après-midi"
				%\EndIf
			%\EndModule
		%\end{Pseudocode}

		%%Remarquez qu'il y a une différence importante entre les
		%%objets et les types de bases. Lorsqu'on déclare une
		%%variable d'un type de base, cela alloue
		%%automatiquement un espace mémoire pour cette variable.
		%%C'est différent avec les objets. La déclaration
		%%n'entraine qu'une réservation mémoire
		%%pour une «~référence~» vers un objet. Celui-ci
		%%n'existe pas encore. Il sera créé (et sa mémoire
		%%allouée) via une instruction spécifique (\pseudocode{nouveau}). On
		%%parle de variable «\textit{~dynamique~}». Le nom est alors~une
		%%«~référence~» vers l’objet. Les avantages de cette dissociation seront
		%%évidents lorsque nous parlerons de la notion de \textit{constructeur}.

		%%Après la déclaration, on a~:
		%%\begin{center}
		%%\begin{tabular}{m{2.2089999cm}}
		%%\centering\arraybslash  midi\\\hline
		%%\multicolumn{1}{|m{2.2089999cm}|}{\centering\arraybslash
		%%\itshape rien}\\\hline
		%%\end{tabular}
		%%\end{center}

		%%Après l'instanciation (ou création), on a~:
		%%\begin{center}
		%%\begin{tabular}{m{2.578cm}m{2.261cm}|m{3.162cm}|}
		%%\centering  midi &
		%%\multicolumn{1}{m{2.261cm}}{~
		%%} &
		%%\multicolumn{1}{m{3.162cm}}{\centering\arraybslash
		 %%Moment}\\\hhline{-~-}
		%%\multicolumn{1}{|m{2.578cm}|}{~
		%%} &
		%%\centering \sffamily $\rightarrow$ &
		%%\centering\arraybslash  heure = ?\\\hhline{-~~}
		%%~
		 %%&
		%%~
		 %%&
		%%\centering\arraybslash  minute = ?\\
		%%~
		 %%&
		%%~
		 %%&
		%%\centering\arraybslash  seconde = ?\\\hhline{~~-}
		%%\end{tabular}
		%%\end{center}
		
		%%Remarquez qu'il n'y a pas d'initialisation par défaut, pour le moment.

		%%Après l'action des mutateurs, on a~:
		%%\begin{center}
		%%\begin{tabular}{m{2.578cm}m{2.261cm}|m{3.162cm}|}
		%%\centering  midi &
		%%\multicolumn{1}{m{2.261cm}}{~
		%%} &
		%%\multicolumn{1}{m{3.162cm}}{\centering\arraybslash
		 %%Moment}\\\hhline{-~-}
		%%\multicolumn{1}{|m{2.578cm}|}{~
		%%} &
		%%\centering \sffamily $\rightarrow$ &
		%%\centering\arraybslash  heure = 12\\\hhline{-~~}
		%%~
		 %%&
		%%~
		 %%&
		%%\centering\arraybslash  minute = 0\\
		%%~
		 %%&
		%%~
		 %%&
		%%\centering\arraybslash  seconde = 0\\\hhline{~~-}
		%%\end{tabular}
		%%\end{center}

		%\begin{Emphase}{Exercices – classe et instance}
			%\vskip-\baselineskip
			%\marginicon{exercice}
			%\begin{enumerate}
				%\item 
					%Pour les produits, vous avez déjà écrit les attributs et les en-têtes des
					%méthodes. Regroupez le tout en une classe \pseudocode{Produit}
					%\textbf{en respectant les notations que vous venez de voir.}
				%\item 
					%Écrivez un module qui affiche le prix d'achat d'un produit, son prix 
					%de vente hors TVA et son prix de vente TVA comprise.
			%\end{enumerate}		
		%\end{Emphase}

%%==========================
%\section{Les constructeurs}
%%===========================

	%L'encapsulation nous permet de contrôler
	%l'état de l'objet et de
	%l'empêcher de tomber dans un état invalide. Mais
	%qu'en est-il de l'état de départ ?
	%Est-il valide ?
	
	%Il serait bon, lorsqu'on crée un objet (via
	%\pseudocode{nouveau}) de pouvoir indiquer l'état
	%initial de l'objet et que cet état puisse être validé.
	%C'est le rôle précis des constructeurs.
	
	%\marginicon{definition}
	%Un \textbf{constructeur} est une méthode particulière permettant
	%d'initialiser les attributs d'un
	%objet lors de sa création effective. Elle porte le même nom que sa
	%classe et ne retourne pas de valeur.
	
	%Il peut y avoir plusieurs constructeurs ce qui permet
	%d'offrir plusieurs possibilités d'indiquer l'état initial de
	%l'objet. Toutefois, nous limiterons au maximum le nombre 
	%de constructeurs dans une classe.
	
	%Remarquez que cela demande de définir plusieurs méthodes 
	%qui portent le même nom.
	
	%\marginicon{definition}
	
	%\textbf{Surcharge} : 
	%le fait de définir plusieurs méthodes portant le même nom.
	%On doit pouvoir les différencier via leurs paramètres.

	%\textbf{Exemple}~:
	%Écrivons des constructeurs pour un moment de la journée~:

	%\begin{Pseudocode}
		%\Class{Moment}
			%\Private
				%\LComment pas de changement
				%\Decl heure~: entier
				%\Decl minute~: entier
				%\Decl seconde~: entier
			%\Public
				%\ConstrSign{Moment}{uneHeure, uneMinute, uneSeconde~: entiers}
				%\ConstrSign{Moment}{uneHeure, uneMinute~: entiers}
				%\RComment 0 seconde par défaut
				%\ConstrSign{Moment}{uneHeure~: entier}
				%\RComment initialiser à une heure pile
				%\Empty
				%\LComment pas de changement au niveau des méthodes~:
				%\MethodSign{getHeure}{}{entier}
				%\MethodSign{getMinute}{}{entier}
				%\MethodSign{getSeconde}{}{entier}
				%\MethodSign{setHeure}{uneHeure~: entier}{}
				%\MethodSign{setMinute}{uneMinute~: entier}{}
				%\MethodSign{setSeconde}{uneSeconde~: entier}{}
				%\MethodSign{estMatin}{}{booléen}
		%\EndClass
		%\end{Pseudocode}

	%\begin{Pseudocode}
		%\Constr{Moment}{uneHeure, uneMinute, uneSeconde~: entiers}
			%\Stmt setHeure(uneHeure)
			%\Stmt setMinute(uneMinute)
			%\Stmt setSeconde(uneSeconde)
		%\EndConstr
		
		%\Empty
		%\Constr{Moment}{uneHeure, uneMinute~: entiers}
			%\Stmt setHeure(uneHeure)
			%\Stmt setMinute(uneMinute)
			%\Stmt setSeconde(0)
		%\EndConstr
		
		%\Empty
		%\Constr{Moment}{uneHeure~: entier}
			%\Stmt setHeure(uneHeure)
			%\Stmt setMinute(0)
			%\Stmt setSeconde(0)
		%\EndConstr
			
		%\Empty
		%\LComment + les accesseurs, les mutateurs et les autres méthodes
	
	%\end{Pseudocode}

	%Contrairement à ce qu'on peut trouver dans certains langages, 
	%comme Java par exemple, nous n'autorisons pas ici d'appel 
	%d'un constructeur d'une classe \pseudocode{A} dans un autre constructeur 
	%de cette même classe \pseudocode{A}.
	
	%Par contre, il est courant en logique qu'un constructeur 
	%appelle les mutateurs afin d'effectuer les tests sans 
	%avoir à les dupliquer.
	%Mais c'est une démarche que vous éviterez de faire dans
	%des langages comme Java par exemple 
	%(cela vous sera expliqué plus tard).
	
	%Lorsqu'on instancie un objet, les paramètres
	%qu'on donne déterminent le constructeur qui est
	%effectivement utilisé pour initialiser l'état de
	%l'objet.

	%\textbf{Exemple}~: Instancions quelques moments de la journée.
	
	%\begin{Pseudocode}
		%\Let heureDépart \Gets \K{nouveau} Moment(14, 23, 56)
		%\Let heureLever \Gets \K{nouveau} Moment(9, 30)
		%\Let heureGouter \Gets \K{nouveau} Moment(17)
	%\end{Pseudocode}

	%Le fait qu'un objet est instancié via la primitive
	%\pseudocode{nouveau} et pas implicitement à la déclaration permet de
	%postposer sa construction effective au moment où
	%l'état initial qu'on veut lui donner
	%sera connu (ce qui peut résulter d'un calcul). On est
	%ainsi assuré que tous les objets manipulés sont valides ce qui permet
	%d’éviter les situations où une méthode fait des dégâts suite à la
	%manipulation d’un objet invalide.
	
	%\begin{Emphase}{Exercices - constructeur}
		%\vskip-\baselineskip
		%\marginicon{exercice}
		%\begin{enumerate}
			%\item {
				%Écrivez un ou des constructeur(s) pour un \pseudocode{Produit}}
			%\item {
				%Adaptez le module écrit plus haut pour qu'il affiche le prix hors TVA
				%puis le prix TVA comprise du produit numéro 105176 (Lego réveil figurine policier)
				%au prix d'achat de 25\euro, au prix de vente de 30\euro et 
				%dont il y a 10 exemplaires en stock.}
		%\end{enumerate}
	%\end{Emphase}

%%================================================
%\section{Du choix de la représentation de l'état}
%%=================================================

	%Lorsqu'on définit une classe, il faut choisir les
	%attributs qui vont permettre de représenter l'état des
	%objets. Cela peut paraitre immédiat mais il n'en est
	%rien.
	
	%\textbf{Exemple}
		
	%Pour un moment de la journée, nous avons choisi
	%d'utiliser trois attributs entiers (les heures, les
	%minutes et les secondes). Nous aurions tout aussi bien pu choisir
	%d'utiliser un seul entier représentant le nombre de
	%secondes écoulées depuis minuit.
	
	%Ces deux représentations sont tout-à-fait équivalentes en terme de
	%potentiel mais la grande différence est l'efficacité
	%du code des méthodes. 
	
	%Prenons deux méthodes symptomatiques~: celle qui donne
	%l'heure et celle qui compare deux moments de la
	%journée. La première est beaucoup plus simple à écrire et plus rapide
	%avec la première représentation alors que la seconde méthode est plus
	%simple à écrire et plus rapide avec la seconde représentation.
	
	%Dès lors, quelle représentation choisir ? Il faut examiner, pour chaque
	%représentation possible, le nombre de méthodes qui sont efficaces mais
	%aussi imaginer la fréquence de leur utilisation (ce qui est difficile
	%et changeant). Heureusement, ce choix n'est pas
	%définitif. Si on change d'avis, on peut changer la
	%représentation. Il faudra bien sûr réécrire les méthodes de la classe
	%mais il ne faudra rien changer au reste du code, c-à-d les lignes du
	%code utilisant la classe. C’est d’ailleurs là une des grandes forces de
	%la programmation orientée objet.

	%\begin{Emphase}{Exercices – représentation de l'état}
		%\vskip-\baselineskip
		%\marginicon{exercice}
		%\begin{enumerate}
			%\item 
				%Compléter la classe Moment en écrivant la méthode «~getHeure~» et celle
				%qui compare deux moments pour les deux représentations imaginées
				%ci-dessus.
			%\item 
				%Écrire le module qui crée deux moments de la journée et vérifie si le
				%premier est avant le second. Ce code dépend-il des attributs choisis
				%pour définir la classe Moment ?
		%\end{enumerate}
	%\end{Emphase}

	%\marginicon{attention}
	%\textbf{Remarque}
	
	%Précédemment, nous avons défini un \textbf{accesseur} comme une méthode
	%permettant d’accéder à la valeur d’un attribut. Mais c’est au
	%développeur de définir quels sont les attributs ; c’est totalement caché
	%à l’utilisateur de la classe. On voit donc bien que cette notion
	%d’accesseur n’a pleinement de sens qu’en interne, pour le développeur
	%de la classe. Pour l’utilisateur il s’agit d’une méthode comme les
	%autres.

	
%%===============================================	
%\section{Représentation modélisée d'une classe}
%%===============================================

	%Un dessin étant souvent plus lisible qu'un texte, on
	%peut représenter graphiquement une classe. Une notation courante est
	%celle utilisée en UML\footnote{{Unified
	%Modeling Langage. }On vous en parlera plus en détail au cours
	%d'Analyse.}. Pour faire simple, une classe est
	%représentée par un rectangle composé de 3 zones~: la première pour le
	%nom de la classe, la deuxième pour les attributs et la troisième pour
	%les méthodes. On indique par un signe «~+~» (resp. «~-~») que le membre
	%est public (resp. privé)

	%\textbf{Exemple}
	
	%\begin{center}
	%\begin{tabular}{|m{7cm}|}
	%\hline
	%\centering\arraybslash \bfseries Moment\\\hline
	%{ {}- heure~: entier}
	
	%{ {}- minute~: entier}
	
	 %{}- seconde~: entier\\\hline
	%{ + getHeure() \textsf{$\rightarrow$} entier}
	
	%{ + setHeure( uneHeure~: entier )}
	
	%{ + avancer1Heure()}
	
	 %...\\\hline
	%\end{tabular}
	%\end{center}

	%Remarquons qu'on indique l'entête des méthodes mais
	%pas le code associé. En fonction du niveau de détail désiré, on
	%pourrait aussi omettre les paramètres et types de retour.


%%%===============================
%%\section{Ce qu'on n'a pas vu...}
%%%================================

	%%Nous n'avons évidemment pas épuisé le sujet de
	%%l'orienté objet. 
	%%Celui-ci est composé de trois piliers~: l'\textbf{encapsulation},
	%%l'\textbf{héritage} et le \textbf{polymorphisme}.
	
	%%Nous venons de voir l'essentiel de la partie «~encapsulation~». 
	%%Les deux autres notions, plus complexes, 
	%%seront abordées au cours de Java 
	%%mais également au cours de Logique de deuxième année. 
	%%En voici un aperçu~:
	
	%%\subsection{L'héritage}
	%%%-----------------------
	
		%%\marginicon{definition}
		%%L’\textbf{héritage} permet de définir une classe 
		%%à partir d’une autre qui lui sert de base. 
		%%On peut alors
		%%\begin{liste}
			%%\item étendre son état ;
			%%\item augmenter ou modifier son comportement.
		%%\end{liste}
		
		%%Par exemple, on pourra définir un \pseudocode{Etudiant} à partir
		%%de la notion de \pseudocode{Personne}.
		
		%%La classe qui sert de point de départ est appelée \textbf{classe de
		%%base}, \textbf{classe mère} ou encore \textbf{super-classe}. La classe
		%%qu’on définit à partir d’une classe mère est appelée \textbf{classe
		%%dérivée}, \textbf{classe fille} ou encore \textbf{sous-classe}.
		
		%%Cette notion est un préalable pour le pilier suivant, le
		%%\textbf{polymorphisme}.
	
	%%\subsection{Le polymorphisme}
	%%%-----------------------------
	
		%%\marginicon{definition}
		%%Le \textbf{polymorphisme} permet d’utiliser un objet fille 
		%%en lieu et place d’un objet mère. 
		
		%%Exposé aussi brièvement cela peut paraitre futile mais cela
		%%permet de construire du code ayant une architecture élégante, robuste
		%%et facilement adaptable.

%%==================
%\section{Exercices}
%%==================

	%Vous trouverez en annexe, 
	%quelques classes que vous pouvez utiliser sans les définir : 
	%\pseudocode{Date}, \pseudocode{Moment} et \pseudocode{Durée}
	
%%\begin{Exercice}{Un produit}
	%%Reprendre la classe \pseudocode{Produit} qui a servi
	%%d'exemple. Identifier et écrire les méthodes qui vous
	%%paraissent utiles pour une telle classe.
	
	%%À partir de là, écrire la classe Stock. Qu'utiliseriez-vous 
	%%comme attributs et quelles méthodes vous paraissent utiles pour
	%%cette classe ?
%%\end{Exercice}

%\begin{Exercice}{Une personne}
	%\marginicon{java}
	%Créer une classe \pseudocode{Personne}, une personne étant
	%constituée d'un nom, d'un prénom et
	%d'une date de naissance. Cette classe utilisera la
	%classe \pseudocode{Date}.

	%On doit pouvoir construire une personne~:

	%\begin{liste}
		%\item 
			%avec 3 arguments~: le nom (\pseudocode{chaine}), le prénom
			%(\pseudocode{chaine}) et la date de naissance de la personne
			%(\pseudocode{Date})
		%\item 
			%avec 2 arguments de type chaine~: le nom et le prénom de la personne; la
			%date de naissance est alors initialisée à «\pseudocode{~rien~}»
	%\end{liste}

	%Écrire aussi tous les accesseurs et mutateurs que vous jugez
	%pertinents. Dans un module principal, créer une personne~:

	%\begin{enumerate}[label=\alph*)]
		%\item 
			%avec comme arguments {\textquotedbl}Durant{\textquotedbl} et
			%{\textquotedbl}Zébulon{\textquotedbl}
		%\item 
			%avec comme arguments {\textquotedbl}Durant{\textquotedbl},
			%{\textquotedbl}Zébulon{\textquotedbl} et la date de naissance du
			%$1^{er}$ février 1989
	%\end{enumerate}

	%Pour réaliser les constructeurs recevant la date de naissance en
	%paramètre, il faudra tester si cette date n’est pas antérieure à la
	%date du jour. 
%\end{Exercice}

%\begin{Exercice}{Anniversaire des personnes}
	%\marginicon{java}
	%À l'aide de la classe \pseudocode{Personne} écrite plus haut,
	%écrire un module qui lit des \pseudocode{Personne} (au clavier)
	%et affiche les noms et le nombre de celles nées ce mois-ci. On suppose
	%que la lecture de «~rien~» indique la fin des données.
%\end{Exercice}

%\begin{Exercice}{La classe rectangle (version orientée objet)}
	%Nous avons déjà abordé les rectangles dans le chapitre des variables
	%structurées. Nous reprenons cet exercice sous l’angle de l’orienté
	%objet.

	%Créer une classe \textbf{Rectangle}
	%permettant de définir des rectangles dont les cotés sont parallèles aux
	%axes des coordonnées dans un plan cartésien. Plusieurs représentations
	%sont possibles~:

	%\begin{liste}
		%\item 
			%la position d’un des sommets et les mesures des cotés;
		%\item 
			%les positions de deux sommets opposés;
		%\item 
			%la position du centre et les demi-mesures des cotés, etc.
	%\end{liste}

	%N’hésitez pas à utiliser la structure Point définissant un point dans un
	%plan cartésien.

	%Une fois les attributs choisis, écrire divers constructeurs~:

	%\begin{enumerate}[label=\alph*)]
		%\item 
			%sans arguments~: le rectangle est un carré de coté 1 centré en (0,0)
		%\item
			%avec deux paramètres~: les mesures des cotés horizontaux et verticaux,
			%le rectangle étant centré en (0,0)
		%\item 
			%avec trois paramètres~: la position du coin en haut à gauche (structure
			%Point) et les mesures des cotés horizontaux et verticaux 
		%\item 
			%avec deux paramètres de type Point~: les positions de deux sommets
			%opposés
	%\end{enumerate}

	%Veillez à vérifier la validité des paramètres !

	%Doter ensuite la classe de méthodes permettant~:

	%\begin{enumerate}[label=\alph*)]
		%\item 
			%d’obtenir la position du centre
		%\item 
			%d’obtenir la position du coin inférieur droit
		%\item 
			%de calculer le périmètre du rectangle
		%\item 
			%de calculer la surface du rectangle
		%\item 
			%de déplacer le rectangle en donnant l’amplitude du déplacement au niveau
			%des abscisses et des ordonnées
		%\item 
			%de multiplier les dimensions du rectangle par un facteur $k$, le centre
			%restant au même endroit
		%\item 
			%de faire pivoter le rectangle de 90 degrés autour de son centre
		%\item 
			%de vérifier si un rectangle à une intersection avec un autre rectangle
	%\end{enumerate}
	
%\end{Exercice}

%\begin{Exercice}{Test}

	%Test~: écrire un module TestRectangle en vue de tester le bon
	%fonctionnement de la classe \pseudocode{Rectangle} de l'exercice précédent. Ce module~: 

	%\begin{enumerate}[label=\alph*)]
		%\item 
			%crée un rectangle R1 par défaut
		%\item 
			%crée un rectangle R2 de cotés 5 et 7, et centré en (0,0)
		%\item 
			%crée un rectangle R3 possédant les sommets (-2,3) et (4, -5)
		%\item 
			%affiche la surface de R1 et le périmètre de R2
		%\item 
			%déplace R1 d’une unité vers le bas
		%\item 
			%déplace R2 de 2 unités vers la droite
		%\item 
			%grossit R3 d’un facteur 3
		%\item 
			%effectue une rotation de 90 degrés à R2
		%\item 
			%indique si R2 et R3 possèdent une intersection
	%\end{enumerate}

	%À présent, choisir une autre représentation des attributs et récrire
	%tout le contenu de la classe (très long, courage !).
	%Récrire ensuite le
	%module TestRectangle (très rapide !)
%\end{Exercice}
