% ====================================
\chapter{Les tableaux à 2 dimensions}
% ====================================

% ===================
\section{Définition}
% ===================

	\marginicon{definition}
	La \textbf{dimension} d’un tableau est le nombre d’indices qu’on utilise
	pour faire référence à un de ses éléments. Attention de ne pas confondre 
	avec la	taille !
	
	En DEV$_1$, nous avons introduit les tableaux à une dimension.
	Un seul indice suffisait à localiser un de ses éléments. 
	Pour le dire autrement,
	chaque case possédait \textbf{un} numéro.
	De nombreuses situations nécessitent cependant l’usage de tableaux à deux dimensions.
	Ils vous sont déjà familiers par leur présence dans beaucoup de
	situations courantes : calendrier, grille horaire, grille de mots
	croisés, sudoku, jeux se déroulant sur un quadrillage (damier,
	échiquier, scrabble\dots).
	Dans ces situations, 
	chaque case est désignée par \textbf{deux} numéros.

% ====================
\section{Notations}
% ====================

	\subsection{Déclarer}
	%====================
		
		\marginicon{definition}
		Pour \textbf{déclarer} un tableau à 2 dimensions, on écrira :
	
		\begin{LDA}
			\Decl{nomTableau}{\Array{nbLignes $\times$ nbColonnes}{TypeElément}}
		\end{LDA}
	
		où \lda{nbLignes} et \lda{nbColonnes} 
		sont des expressions entières quelconques.
	
		\textbf{Exemple} :
		
		\begin{LDA}
			\Decl{tab}{\Array{5$\times$10}{entiers}}
		\end{LDA}
		
		déclare un tableau de 5 lignes par 10 colonnes dont chaque case
		contient un entier.
		
	\subsection{Utiliser}
	%====================

		\marginicon{definition}
		Pour \textbf{accéder} à une case du tableau on donnera les deux indices
		entre crochets.
		Comme en DEV$_1$, on va considérer que la première ligne 
		et la première colonne portent le numéro (l'indice) $0$.  
		
		\textbf{Exemple} :
		
		\begin{LDA}
			\Write tab[2,4] 
				\RComment{affiche le 5\ieme{} élément de la 3\ieme{} ligne du tableau nommé \lda{tab}.}
		\end{LDA}
		
	\subsection{Visualiser}
	%======================

		Notez que la vue sous forme de tableau avec des lignes
		et des colonnes est une vision humaine.
		Il n'y a pas de lignes ni de colonnes en mémoire.
		Pour être précis,
		on devrait juste parler de première dimension et
		de deuxième dimension
		mais la notion de ligne et de colonne
		est un abus de langage qui simplifie le discours.
		
		On pourrait aussi visualiser un tableau à deux dimensions
		comme un tableau à une dimension dont chacun des éléments est
		lui-même un tableau à une dimension.
	
		\textbf{Exemple} : Soit le tableau déclaré ainsi :
	
		\begin{LDA}
		\Decl{tabLettres}{\Array{4$\times$5}{caractères}}
		\end{LDA}
	
		On peut le visualiser à l’aide d’une grille à 4 lignes et 5 colonnes.

		\begin{center}
		\begin{tabular}{*{6}{>{\centering\arraybackslash}m{3mm}}}
			{} & 0 & 1 & 2 & 3 & 4 \\
		\end{tabular}
		\\
		\begin{tabular}{>{\centering\arraybackslash}m{3mm}|*{5}{>{\centering\arraybackslash}m{3mm}|}}
			\cline{2-6}
			0 & d & h & v & q & z \\\cline{2-6}
			1 & j & g & k & o & u \\\cline{2-6}
			2 & i & f & y & r & t \\\cline{2-6}
			3 & n & d & e & a & s \\\cline{2-6}
		\end{tabular}
		\end{center}
	
		Ainsi, la valeur de \lda{tabLettres[2,3]} 
		est le caractère ‘r’. 
		
		La vision «~tableau de tableaux~» 
		(ou décomposition en niveaux)
		donnerait :
	
		\begin{center}
		\begin{tabular}{*{4}{>{\centering\arraybackslash}m{30mm}}}
			0 & 1 & 2 & 3 \\
		\end{tabular}
		\\
		\begin{tabular}{|*{4}{>{\centering\arraybackslash}m{30mm}|}}
			\hline
			\begin{minipage}{30mm}
				\begin{tabular}{*{5}{>{\centering\arraybackslash}m{1.5mm}}}
					0 & 1 & 2 & 3 & 4 \\
				\end{tabular}
				\\		
				\begin{tabular}{|*{5}{>{\centering\arraybackslash}m{1.5mm}|}}
					\hline
					d & h & v & q & z \\
					\hline
				\end{tabular}
				\\
			\end{minipage}
			& 
			\begin{minipage}{30mm}
				\begin{tabular}{*{5}{>{\centering\arraybackslash}m{1.5mm}}}
					0 & 1 & 2 & 3 & 4 \\
				\end{tabular}
				\\		
				\begin{tabular}{|*{5}{>{\centering\arraybackslash}m{1.5mm}|}}
					\hline
					j & g & k & o & u \\
					\hline
				\end{tabular}
				\\
			\end{minipage}
			& 
			\begin{minipage}{30mm}
				\begin{tabular}{*{5}{>{\centering\arraybackslash}m{1.5mm}}}
					0 & 1 & 2 & 3 & 4 \\
				\end{tabular}
				\\		
				\begin{tabular}{|*{5}{>{\centering\arraybackslash}m{1.5mm}|}}
					\hline
					i & f & y & r & t \\
					\hline
				\end{tabular}
				\\
			\end{minipage}
			& 
			\begin{minipage}{30mm}
				\begin{tabular}{*{5}{>{\centering\arraybackslash}m{1.5mm}}}
					0 & 1 & 2 & 3 & 4 \\
				\end{tabular}
				\\		
				\begin{tabular}{|*{5}{>{\centering\arraybackslash}m{1.5mm}|}}
					\hline
					n & d & e & a & s \\
					\hline
				\end{tabular}
				\\
			\end{minipage}
			\\
			\hline
		\end{tabular}
		\end{center}

		Dans cette représentation, le tableau \lda{tabLettres} est
		d’abord décomposé à un premier niveau en quatre éléments auxquels on
		accède par le premier indice. Ensuite, chaque élément de premier niveau
		est décomposé en cinq éléments de deuxième niveau accessibles par le
		deuxième indice.

	\subsection{Exemples}
	%======================
	
		\paragraph{Exemple 1 -- Remplir les coins.}
		Dans ce petit exemple,
		on a un tableau de chaines et on donne des valeurs aux coins.
		
		\begin{center}
			\begin{tabular}{|*{5}{>{\centering\arraybackslash}m{1cm}|}}
				\hline
				"NO" &
				 &
				 &
				 &
				"NE"
				\\\hline
				 &
				 &
				 &
				 &
				\\\hline
				"SO" &
				 &
				 &
				 &
				"SE"
				\\\hline
			\end{tabular}	
		\end{center}
		
		\begin{LDA}
		\LComment Déclare un tableau et donne des valeurs aux coins.
		\Algo{remplirCoins}{}{}
			\Decl{grille}{\Array{3$\times$5}{entiers}}
			\Let grille[0,0] \Gets "NO"
			\Let grille[0,4] \Gets "NE"
			\Let grille[2,0] \Gets "SO"
			\Let grille[2,4] \Gets "SE"
		\EndAlgo
		\end{LDA}
	
		\paragraph{Exemple 2 -- Gestion des stocks.}
		Reprenons l'exemple du stock de 10 produits
		qui a servi d'introduction au chapitre sur les tableaux
		mais, cette fois, pour chaque jour de la semaine.
	
		\begin{small}
		\begin{center}
			%\begin{tabular}{*{8}{>{\centering\arraybackslash}m{1.2cm}}}
			\begin{tabular}{m{1.5cm}*{7}{>{\centering\arraybackslash}m{1.2cm}}}
				~ & {article0} & {article1} & {article2} & \dots & {article7} & {article8} & {article9}\\
			\end{tabular}	
			\begin{tabular}{|m{1.5cm}|*{7}{>{\centering\arraybackslash}m{1.2cm}|}}
				\hline
				{lundi}    & {cpt[0,0]} & {cpt[0,1]} & {cpt[0,2]} & \dots & {cpt[0,7]} & {cpt[0,8]} & {cpt[0,9]} \\\hline
				{mardi}    & {cpt[1,0]} & {cpt[1,1]} & {cpt[1,2]} & \dots & {cpt[1,7]} & {cpt[1,8]} & {cpt[1,9]} \\\hline
				{mercredi} & {cpt[2,0]} & {cpt[2,1]} & {cpt[2,2]} & \dots & {cpt[2,7]} & {cpt[2,8]} & {cpt[2,9]} \\\hline
				{jeudi}    & {cpt[3,0]} & {cpt[3,1]} & {cpt[3,2]} & \dots & {cpt[3,7]} & {cpt[3,8]} & {cpt[3,9]} \\\hline
				{vendredi} & {cpt[4,0]} & {cpt[4,1]} & {cpt[4,2]} & \dots & {cpt[4,7]} & {cpt[4,8]} & {cpt[4,9]} \\\hline
				{samedi}   & {cpt[5,0]} & {cpt[5,1]} & {cpt[5,2]} & \dots & {cpt[5,7]} & {cpt[5,8]} & {cpt[5,9]} \\\hline
				{dimanche} & {cpt[6,0]} & {cpt[6,1]} & {cpt[6,2]} & \dots & {cpt[6,7]} & {cpt[6,8]} & {cpt[6,9]} \\\hline
			\end{tabular}
		\end{center}
		\end{small}
		
		\bigskip
		
		\begin{LDA}
		\LComment Calcule et affiche la quantité vendue de 10 produits
		\LComment pour chaque jour de la semaine (de 0~: lundi à 6~: dimanche).
		\Algo{statistiquesVentesSemaine}{}{}
			\Empty
			\Decl{cpt}{\Array{7$\times$10}{entiers}}
			%\Decl{produit, jour}{entiers}
			\Empty
			\Stmt initialiser(cpt)
			\Empty
			\LComment Pour chaque jour de la semaine
			\For{jour}{0}{6}
				\Stmt traiterStock1Jour(cpt, jour)
				\For{produit}{0}{9}
					\Write "quantité vendue de produit ", produit, " ce jour ", jour, "~: ", cpt[jour,produit]
				\EndFor
			\EndFor	
		\EndAlgo
		\end{LDA}
	
		\begin{LDA}
		\LComment Initialise le tableau d'entiers à 0
		\Algo{initialiser}{\Par{entiers\InOut}{\Array{7$\times$10}{entiers}}}{}
			%\Decl{i, j}{entiers}
			\For{i}{0}{6}
				\For{j}{0}{9}
					\Let entiers[i,j] \Gets 0
				\EndFor
			\EndFor
		\EndAlgo
		\end{LDA}
	
		\begin{LDA}
		\LComment Effectue le traitement du stock pour une journée.
		\Algo{traiterStock1Jour}{\Par{cpt\InOut}{\Array{7$\times$10}{entiers}}, \Par{jour~}{entier}}{}
			\Decl{numéroProduit, quantité}{entiers}
			\Write "Introduisez le numéro du produit~:"
			\Read numéroProduit
			\Empty
			\While{numéroProduit $\geq$ 0}
				\Empty
				\Write "Introduisez la quantité vendue~:"
				\Read quantité
				\Empty
				\Let cpt[jour,numéroProduit] \Gets cpt[jour,numéroProduit] + quantité
				\Empty
				\Write "Introduisez le numéro du produit~:"
				\Read numéroProduit
				\Empty
			\EndWhile
		\EndAlgo
		\end{LDA}
		
		Pour plus d'exemples, allez faire un tour à la section \vref{algo:Tab2D}.
	
% ============================================
\section{La troisième dimension (et au-delà)}
% ============================================

	Certaines situations complexes nécessitent l'usage de
	tableaux à 3 voire plus de dimensions.

	\marginicon{definition}
	Pour déclarer un tableau statique à $k$ dimensions, on écrira :

	\begin{LDA}
	\Decl{nomTableau}{\Array{tailleDim1 $\times$ \dots{} $\times$ tailleDimK}{TypeElément}}
	\end{LDA}
	
	
% ================================================
\section{Parcours d'un tableau à deux dimensions}
\label{algo:Tab2D}
% ================================================

	Comme nous l'avons fait pour les tableaux à  une dimension,
	envisageons le parcours des tableaux à deux dimensions 
	($n$ lignes et $m$ colonnes).
	Nos algorithmes sont valables quel que soit le type des éléments.
	Utilisons \lda{T} pour désigner un type quelconque.
	
	\begin{LDA}
		\Decl{tab}{\Array{n $\times$ m}{T}}
	\end{LDA}
	
	Commençons par des cas plus simples 
	où on ne parcourt qu'une seule des dimensions 
	puis attaquons le cas général.

	\subsection{Parcours d'une dimension}
	%====================================
	
		On peut vouloir ne parcourir qu'une seule ligne du tableau.
		Si on parcourt la ligne $l$, on visite les cases 
		$(l,0)$, $(l,1)$, \dots, $(l,m-1)$.
		L'indice de ligne est constant et c'est l'indice de colonne qui varie.
		
		\begin{center}
		$l$
		\begin{tabular}{|*{5}{>{\centering\arraybackslash}m{0.3cm}|}}
		\hline
		\ & \ & \ & \ & \  \\
		\hline
		\cellcolor{gray!25}\ & \cellcolor{gray!25}\ & \cellcolor{gray!25}\ & \cellcolor{gray!25}\ & \cellcolor{gray!25}\  \\
		\hline
		\ & \ & \ & \ & \  \\
		\hline
		\end{tabular}
		\end{center}
		
		Ce qui donne l'algorithme :
		
		\begin{LDA}
			\LComment{Parcours de la ligne $ligne$ d'un tableau à deux dimensions}
			\Algo{affichageElémentsLigne}{\Par{tab}{\Array{n $\times$ m}{T}}, 
				\Par{ligne}{entier}}{}
				\For{colonne}{0}{m-1}
					\Write tab[ligne,colonne] \RComment On peut faire autre chose qu'afficher
				\EndFor
			\EndAlgo
		\end{LDA}

		Retenons~: pour parcourir une ligne, on utilise une boucle sur les colonnes. 
		
		Symétriquement, on pourrait considérer le parcours de la colonne $c$
		avec l'algorithme suivant.
		
		\begin{LDA}
			\LComment{Parcours de la colonne $colonne$ d'un tableau à deux dimensions}
			\Algo{affichageElémentsColonne}{\Par{tab}{\Array{n $\times$ m}{T}}, 
				\Par{colonne}{entier}}{}
				\For{ligne}{0}{n-1}
					\Write tab[ligne,colonne] \RComment On peut faire autre chose qu'afficher
				\EndFor
			\EndAlgo
		\end{LDA}
		
		Si le tableau est carré ($n=m$) on peut aussi envisager le parcours
		des deux diagonales.
		
		Pour la diagonale descendante, 
		les éléments à visiter sont $(0,0)$, $(1,1)$, $(2,2)$, \dots, $(n-1,n-1)$.
		
		\begin{center}
		\begin{tabular}{|*{3}{>{\centering\arraybackslash}m{0.3cm}|}}
		\hline
		\cellcolor{gray!25}\ & \ & \ \\
		\hline
		\ & \cellcolor{gray!25}\ & \ \\
		\hline
		\ & \ & \cellcolor{gray!25}\ \\
		\hline
		\end{tabular}
		\end{center}
		
		Une seule boucle suffit 
		comme le montre l'algorithme suivant.
		
		\begin{LDA}
			\LComment{Parcours de la diagonale descendante d'un tableau carré}
			\Algo{affichageElémentsDiagonaleDescendante}{\Par{tab}{\Array{n $\times$ n}{T}}}{}
				\For{i}{0}{n-1}
					\Write tab[i,i] \RComment On peut faire autre chose qu'afficher
				\EndFor
			\EndAlgo
		\end{LDA}
		
		Pour la diagonale montante, 
		on peut envisager deux solutions, 
		avec deux indices ou un seul
		en se basant sur le fait que $i+j=n-1 \Rightarrow j=n-1-i$.
		
		\begin{center}
		\begin{tabular}{|*{3}{>{\centering\arraybackslash}m{0.3cm}|}}
		\hline
		\ & \ & \cellcolor{gray!25}\ \\
		\hline
		\ & \cellcolor{gray!25}\ & \ \\
		\hline
		\cellcolor{gray!25}\ & \ & \ \\
		\hline
		\end{tabular}
		\end{center}
		
		\begin{LDA}
			\LComment{Parcours de la diagonale montante d'un tableau carré - 2 indices}
			\Algo{affichageElémentsDiagonaleMontante}{\Par{tab}{\Array{n $\times$ n}{T}}}{}
				\Decl {j}{entier}
				\Let j \Gets n-1
				\For{i}{0}{n-1}
					\Write tab[i,j] \RComment On peut faire autre chose qu'afficher
					\Let j \Gets j - 1
				\EndFor
			\EndAlgo
		\end{LDA}
		
		\begin{LDA}
			\LComment{Parcours de la diagonale montante d'un tableau carré - 1 indice}
			\Algo{affichageElémentsDiagonaleMontante}{\Par{tab}{\Array{n $\times$ n}{T}}}{}
				\For{i}{0}{n-1}
					\Write tab[i, n - 1 - i]  \RComment On peut faire autre chose qu'afficher
				\EndFor
			\EndAlgo
		\end{LDA}

	\subsection{Parcours des deux dimensions}
	%========================================
	
		\subsubsection*{Parcours par lignes et par colonnes}
		%---------------------------------------------------
			
			Les deux parcours les plus courants sont les parcours ligne par ligne
			et colonne par colonne.
			Les tableaux suivants montrent dans quel ordre chaque case est visitée dans ces deux parcours.
			
			\begin{center}
			\begin{minipage}{0.4\textwidth}
			\begin{center}
			Parcours ligne par ligne\\
			\begin{tabular}{|*{5}{>{\centering\arraybackslash}m{0.35cm}|}}
			\hline
			1 & 2 & 3 & 4 & 5 \\
			\hline
			6 & 7 & 8 & 9 & 10 \\
			\hline
			11 & 12 & 13 & 14 & 15 \\
			\hline
			\end{tabular}
			\end{center}
			\end{minipage}
			\qquad
			\begin{minipage}{0.4\textwidth}
			\begin{center}
			Parcours colonne par colonne\\
			\begin{tabular}{|*{5}{>{\centering\arraybackslash}m{0.35cm}|}}
			\hline
			1 & 4 & 7 & 10 & 13 \\
			\hline
			2 & 5 & 8 & 11 & 14 \\
			\hline
			3 & 6 & 9 & 12 & 15 \\
			\hline
			\end{tabular}
			\end{center}
			\end{minipage}
			\end{center}
			
			Le plus simple est d'utiliser deux boucles imbriquées 
			
			\begin{LDA}
				\LComment{Parcours d'un tableau à 2 dimensions, ligne par ligne}
				\Algo{affichageElémentsLigneParLigne}{\Par{tab}{\Array{n $\times$ m}{T}}}{}
					\For{lg}{0}{n-1}
						\For{col}{0}{m-1}
							\Write tab[lg,col] \RComment On peut faire autre chose qu'afficher
						\EndFor
					\EndFor
				\EndAlgo
			\end{LDA}
			
			\begin{LDA}
				\LComment{Parcours d'un tableau à 2 dimensions, colonne par colonne}
				\Algo{affichageElémentsColonneParColonne}{\Par{tab}{\Array{n $\times$ m}{T}}}{}
					\For{col}{0}{m-1}
						\For{lg}{0}{n-1}
							\Write tab[lg,col] \RComment On peut faire autre chose qu'afficher
						\EndFor
					\EndFor
				\EndAlgo
			\end{LDA}
			
			Mais on peut obtenir le même résultat avec une seule boucle
			si l'indice sert juste à compter le nombre de passages
			et que les indices de lignes et de colonnes sont gérés manuellement.
			
			L'algorithme suivant montre ce que ça donne
			pour un parcours ligne par ligne.
			La solution pour un parcours colonne par colonne est similaire
			et laissée en exercice.
			
			\begin{LDA}
				\LComment{Parcours d'un tableau à 2 dimensions via une seule boucle}
				\Algo{affichageElémentsLigneParLigne}{\Par{tab}{\Array{n $\times$ m}{T}}}{}
					\Decl {lg, col}{entiers}
					\Let lg \Gets 0
					\Let col \Gets 0
					\For{i}{0}{(n*m)-1}
						\Write tab[lg,col] \RComment On peut faire autre chose qu'afficher
						\Let col \Gets col + 1	\RComment Passer à la case suivante
						\If{col = m} \RComment On déborde sur la droite, passer à la ligne suivante
							\Let col \Gets 0
							\Let lg \Gets lg + 1
						\EndIf
					\EndFor
				\EndAlgo
			\end{LDA}
			
			L'avantage de cette solution apparaitra 
			quand on verra des situations plus difficiles.

		\subsubsection*{Interrompre le parcours}
		%---------------------------------------
			
			Comme avec les tableaux à une dimension, 
			envisageons l'arrêt prématuré lors de la rencontre d'une certaine condition.
			Et, comme avec les tableaux à une dimension, 
			transformons d'abord nos \K{pour} en \K{tant que}.
			
			Par exemple, montrons les deux parcours ligne par ligne, avec une et deux boucle(s).
			
			\begin{LDA}
				\LComment{Parcours d'un tableau à 2 dimensions, ligne par ligne, via un tant que}
				\Algo{affichageElémentsLigneParLigne}{\Par{tab}{\Array{n $\times$ m}{T}}}{}
					\Decl {lg, col}{entiers}
					\Let lg \Gets 0
					\While{lg $<$ n}
						\Let col \Gets 0
						\While{col $<$ m}
							\Write tab[lg, col] \RComment On peut faire autre chose qu'afficher
							\Let col \Gets col + 1
						\EndWhile
						\Let lg \Gets lg + 1
					\EndWhile
				\EndAlgo
			\end{LDA}
			
			\begin{LDA}
				\LComment{Parcours d'un tableau à 2 dimensions via une seule boucle et un tant que}
				\Algo{affichageElémentsLigneParLigne}{\Par{tab}{\Array{n $\times$ m}{T}}}{}
					\Decl {lg, col, i}{entiers}
					\Let lg \Gets 0
					\Let col \Gets 0
					\Let i \Gets 0
					\While{i $<$ n*m} \RComment ou "lg $<$ n" 
						\Write tab[lg,col] \RComment On peut faire autre chose qu'afficher
						\Let col \Gets col + 1	\RComment Passer à la case suivante
						\If{col = m} \RComment On déborde sur la droite, passer à la ligne suivante
							\Let col \Gets 0
							\Let lg \Gets lg + 1
						\EndIf
						\Let i \Gets i + 1		
					\EndWhile
				\EndAlgo
			\end{LDA}
			
			On peut à présent introduire le test comme on l'a fait 
			dans les algorithmes de parcours des tableaux à une dimension.
			
			Illustrons-le au travers de deux exemples
			où on cherche un élément particulier.
			Le premier introduit un test en utilisant un booléen
			alors que le second introduit un test
			sans utiliser de booléen.
			
			\begin{LDA}
				\LComment{Parcours avec test d'arrêt - deux boucles et un booléen}
				\Algo{chercherEléments}{\Par{tab}{\Array{n $\times$ m}{T}}}{}
					\Decl {lg, col}{entiers}
					\Decl {trouvé}{booléen}
					\Let trouvé \Gets faux
					\Let lg \Gets 0
					\While{lg $<$ n ET NON trouvé}
						\Let col \Gets 0
						\While{col $<$ m ET NON trouvé}
							\If{\textit{tab[lg, col] est l'élément recherché}}
								\Let trouvé \Gets vrai
							\Else \RComment Ne pas modifier les indices si arrêt demandé
								\Let col \Gets col + 1
							\EndIf
						\EndWhile
						\If{NON trouvé} \RComment Ne pas modifier les indices si arrêt demandé
							\Let lg \Gets lg + 1
						\EndIf
					\EndWhile
					\LComment Tester trouvé pour savoir si on a trouvé l'élément recherché
				\EndAlgo
			\end{LDA}
			
			\begin{LDA}
				\LComment{Parcours avec test d'arrêt - une boucle et pas de booléen}
				\Algo{chercherEléments}{\Par{tab}{\Array{n $\times$ m}{T}}}{}
					\Decl {lg, col, i}{entiers}
					\Let lg \Gets 0
					\Let col \Gets 0
					\Let i \Gets 0
					\While{i $<$ n*m ET \textit{tab[lg, col] n'est pas l'élément recherché}}  
						\Let col \Gets col + 1	\RComment Passer à la case suivante
						\If{col = m} \RComment On déborde sur la droite, passer à la ligne suivante
							\Let col \Gets 0
							\Let lg \Gets lg + 1
						\EndIf
						\Let i \Gets i + 1		
					\EndWhile
					\LComment L'élément recherché a été trouvé si i $<$ n*m.
				\EndAlgo
			\end{LDA}

		\subsubsection*{Parcours plus compliqué - le serpent}
		%----------------------------------------------------
			
			Envisageons un parcours plus difficile illustré par le tableau suivant.
			
			\begin{center}
			\begin{tabular}{|*{5}{>{\centering\arraybackslash}m{0.35cm}|}}
			\hline
			1 & 2 & 3 & 4 & 5 \\
			\hline
			10 & 9 & 8 & 7 & 6 \\
			\hline
			11 & 12 & 13 & 14 & 15 \\
			\hline
			\end{tabular}
			\end{center}
			
			Le plus simple est d'adapter l'algorithme de parcours 
			avec une seule boucle
			en introduisant un sens de déplacement, 
			ce qui donne l'algorithme :
			
			\begin{LDA}
				\LComment{Parcours du serpent dans un tableau à deux dimensions}
				\Algo{affichageElémentsSerpent}{\Par{tab}{\Array{n $\times$ m}{T}}}{}
					\Decl {lg, col, depl}{entiers}
					\Let lg \Gets 0
					\Let col \Gets 0
					\Let depl \Gets 1	\RComment 1 pour avancer, -1 pour reculer
					\For{i}{0}{(n*m)-1}
						\Write tab[lg, col] \RComment On peut faire autre chose qu'afficher
						\If{0 $\le$ col + depl ET col + depl $<$ m}
							\Let col \Gets col + depl \RComment On se déplace dans la ligne
						\Else
							\Let lg \Gets lg + 1	\RComment On passe à la ligne suivante
							\Let depl \Gets -depl	\RComment et on change de sens
						\EndIf
					\EndFor
				\EndAlgo
			\end{LDA}

% ===================
\section{Exercices}
% ===================

	\begin{Exercice}{Affichage}
		Écrire un algorithme qui affiche tous les éléments d'un
		tableau à $n$ lignes et $m$ colonnes
		\\\quad a) ligne par ligne \qquad b) colonne par colonne
	\end{Exercice}
	
	\begin{Exercice}{Les nuls}
		\marginicon{java}
		Écrire un algorithme qui reçoit un tableau ($n$ x $m$)
		d'entiers et qui affiche la proportion
		d'éléments nuls dans ce tableau.
	\end{Exercice}
	
	\begin{Exercice}{Tous positifs}
		\marginicon{java}
		Écrire un algorithme qui reçoit un tableau ($n$ x $m$) d’entiers et qui vérifie
		si tous les nombres qu’il contient sont strictement positifs. Bien sûr,
		on veillera à éviter tout travail inutile; la rencontre d’un nombre
		négatif doit arrêter le module.
	\end{Exercice}
	
	\begin{Exercice}{Le tableau de cotes}
		Soit un tableau à $n$ lignes et $m$ colonnes d'entiers où
		une ligne représente les notes sur 20 d'un étudiant et
		les colonnes toutes les notes d'un cours.
		
		Écrire un algorithme recevant ce tableau en paramètre et affichant le
		pourcentage d'étudiants ayant obtenu une moyenne
		supérieure à 50\%.
	\end{Exercice}
	
	\begin{Exercice}{Le carré magique}
		\marginicon{java}
		Un carré magique est un tableau d’entiers carré
		(c'est-à-dire possédant autant de lignes que de
		colonnes) ayant la propriété suivante: si on additionne les éléments
		d'une quelconque de ses lignes, de ses colonnes ou de
		ses deux diagonales, on obtient à chaque fois le même résultat.
	
		Écrire un algorithme recevant en paramètres le tableau ($n$ x $n$)
		d'entiers représentant le carré et renvoyant une valeur booléenne
		indiquant si c'est un carré magique ou pas.
	\end{Exercice}
	
	\begin{Exercice}{Le triangle de Pascal}
		Le triangle de Pascal est construit de la façon suivante :
		\begin{itemize}
		\item la ligne initiale contient un seul élément de valeur 1 ;
		\item chaque ligne possède un élément de plus que la précédente ;
		\item chaque ligne commence et se termine par 1 ;
		\item 
			pour calculer un nombre d’une autre case du tableau, on additionne le
			nombre situé dans la case située juste au-dessus avec celui dans la
			case à la gauche de la précédente.
		\end{itemize}
	
		Écrire un algorithme qui reçoit en paramètre un entier
		$n$, et qui renvoie un tableau contenant les
		$n+1$ premières lignes du triangle de Pascal
		(indicées de $0$ à $n$).
		
		N.B.: le «~triangle~» sera bien entendu renvoyé dans un tableau carré.
		Quid des cases non occupées ?
	
		Par exemple, pour $n$ qui vaut 5, on aura le tableau suivant :
	
		\begin{center}
		\begin{tabular}{|*{6}{>{\centering\arraybackslash}m{0.35cm}|}}
		\hline
		 1 & ~ & ~ & ~ & ~ & ~ \\\hline
		 1 & 1 & ~ & ~ & ~ & ~ \\\hline
		 1 & 2 & 1 & ~ & ~ & ~ \\\hline
		 1 & 3 & 3 & 1 & ~ & ~ \\\hline
		 1 & 4 & 6 & 4 & 1 & ~ \\\hline
		 1 & 5 & 10 & 10 & 5 & 1 \\\hline
		\end{tabular}
		\end{center}
	\end{Exercice}
	
	\begin{Exercice}{Lignes et colonnes}
		Écrire un algorithme qui reçoit un tableau d’entiers à 2 dimensions en paramètre 
		et qui retourne un booléen indiquant si ce tableau 
		possède 2 lignes ou 2 colonnes identiques.
		
		Dans l’affirmative, 
		cet algorithme renverra également en paramètres les informations suivantes :
		\begin{itemize}
		\item les indices des lignes ou colonnes identiques
		\item un caractère valant ‘L’ ou ‘C’ selon qu’il s’agit de lignes ou de
		colonnes
		\end{itemize}
		
		Dans la négative, les valeurs de ces paramètres seront indéterminées ou
		quelconques, elles ne seront de toute façon pas utilisées par le module
		appelant.
	\end{Exercice}
	
	\begin{Exercice}{Le contour du tableau}
		\marginicon{java}
		On donne un tableau d’entiers \lda{tabEnt} 
		à $n$ lignes et $m$ colonnes. 
		Écrire un algorithme retournant la somme 
		de tous les éléments \textit{impairs}
		situés sur le bord du tableau.
	
		Exemple : pour le tableau suivant, l'algorithme doit renvoyer $32$
	
		\begin{center}
		\begin{tabular}{|*{4}{>{\centering\arraybackslash}m{0.6cm}|}}
		  \hline
		  3 & 4 & 6 & 11\\\hline
		  2 & 21 & 7 & 9\\\hline
		  1 & 5 & 12 & 3\\\hline
		\end{tabular}
		\end{center}
	
		Et pour le suivant, l'algorithme doit renvoyer $6$
	
		\begin{center}
		\begin{tabular}{|*{5}{>{\centering\arraybackslash}m{0.3cm}|}}
		\hline
		 4 & 1 & 2 & 8 & 5\\\hline
		\end{tabular}
		\end{center}
	\end{Exercice}
	
	\begin{Exercice}{À vos pinceaux !}
		On possède un tableau à $n$ lignes et $n$ colonnes dont les éléments de type
		Couleur valent NOIR ou BLANC. On suppose que le tableau est initialisé
		à BLANC au départ. Écrire un algorithme qui \emph{noircit} les cases de ce
		tableau comme le suggèrent les dessins suivants~(les exemples sont
		donnés pour un tableau 10 x 10 mais les algorithmes doivent fonctionner
		quelle que soit la taille du tableau).
		
		\begin{center}
		\includegraphics[width=0.7\textwidth]{image/tab2d-ex-oxz}
		\includegraphics[width=0.7\textwidth]{image/tab2d-ex-zts}
		\end{center}
		
		Notes
		\begin{itemize}
		\item 
			Le zèbre doit toujours présenter des lignes
			obliques et parallèles, quelle que soit la taille.
		\item
			La spirale est un véritable défi
			et vous est donné comme exercice facultatif.
			Ne le faites pas si vous êtes en retard.
		\end{itemize}
		
	\end{Exercice}
	
	\begin{Exercice}{Exercices sur la complexité}
	
		Quelle est la complexité 
		\begin{enumerate}[label=\alph*)]
		\item 
			d’un algorithme de parcours	d'un tableau $n$ x $n$ ?
		\item
			des algorithmes que vous avez écrits pour les exercices :
			"Les nuls", "Tous positifs", "Le carré magique"
			et "Le contour d'un tableau" ?
		\item 
			des algorithmes que vous avez écrits pour résoudre les
			exercices du pinceau ?
		\end{enumerate}
	\end{Exercice}

